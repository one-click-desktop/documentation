\documentclass[../../deliverable-two.tex]{subfiles}

\begin{document}
\label{external-tools}

\subsection{Ansible}
Ansible zostanie wykorzystany w systemie do zaaplikowania zmiennej konfiguracji do każdej uruchamianej wirtualnej maszyny.
Podstawowo playbook będzie zawierać informacje o:
\begin{enumerate}
	\item Dane dostępowe do dysku sieciowego oraz wykorzystany protokół
	\item Dane dostępowe do usługi katalogowej
	\item TODO: dopisac wsyztskie potrzebne konfiguracje
\end{enumerate}
Playbook można rozszerzać o potrzebne dane zależne od użycia.

\subsection{Vagrant}
Vagrant zostanie wykorzystany w celu łatwej parametryzacji oraz powtarzalnego tworzenia maszyn wirtualnych z przygotowanego wcześniej obrazu systemu.
Głównie wykorzystany będzie mechanizm Vagrantboxów, które są obrazami wcześniej przygotowanego systemu operacyjnego.
Aby system działał prawidłowo obraz systemu zamknięty w Vagrantboxie musi spełniać następujące warunki:
\begin{enumerate}
	\item Użytkownicy muszą być pobierani z usługi katalogowej.
	\item Katalogi domowe użytkowników muszą być na dysku sieciowym.
	\item TODO: dopisac wsyztskie potrzebne wymogi
\end{enumerate}

\subsection{Libvirt z QEMU}
Libvirt połączony z QEMU będzie wykorzystany do zarządzania maszynami wirtualnymi uruchamianymi na serwerze wirtualizacji.
Umożliwi on:
\begin{enumerate}
	\item Tworzenie maszyn wirtualnych.
	\item Uruchamianie maszyn wirtualnych.
	\item Przyporządkowanie zasobów maszynom wirtualnym (w tym kraty graficzne).
	\item Wyłączanie maszyn wirtualnych.
	\item Sprawdzanie, czy maszyna działa na serwerze wirtualizacji.
\end{enumerate}

\end{document}