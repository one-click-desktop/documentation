\documentclass[../deliverable-two.tex]{subfiles}

\begin{document}

\begin{itemize}
    \item Aplikacja kliencka
          \begin{itemize}
              \item Typescript\footnote{\href{https://www.typescriptlang.org}{Strona projektu Typescript}}
                    /Javascript\footnote{\href{https://www.ecma-international.org/publications-and-standards/standards/ecma-262}{Obecny standard języka Javascript}}
              \item Node.js\footnote{\href{https://nodejs.org/en}{Strona projektu Node.js}} - środowisko uruchomieniowe używane do integracji z systemem użytkownika
              \item Angular\footnote{\href{https://angular.io/}{Strona projektu Angular}} - renderowanie widoków
              \item Electron\footnote{\href{https://www.electronjs.org/}{Strona projektu Electron}} - platforma programistyczna
              \item Jest\footnote{\href{https://jestjs.io/}{Strona projektu Jest}} - testy jednostkowe
              \item Cypress\footnote{\href{https://www.cypress.io/}{Strona projektu Cypress}} - testy integracyjne
          \end{itemize}
    \item Panel administratora
          \begin{itemize}
              \item Typescript/Javascript
              \item Angular - platforma aplikacji WWW
              \item Jest - testy jednostkowe
              \item Cypress - testy integracyjne
          \end{itemize}
    \item Nadzorca i serwer wirtualizacji
          \begin{itemize}
              \item C\#\footnote{\href{https://docs.microsoft.com/en-us/dotnet/csharp/}{Dokumentacja języka C\#}}
              \item RabbitMQ\footnote{\href{https://www.rabbitmq.com/}{Strona projektu RabbitMQ}} - broker asynchronicznych wiadomości
              \item Ansible\footnote{\href{https://www.ansible.com/}{Strona projektu Ansible}} - konfigurowanie maszyn wirtualnych
              \item Vagrant\footnote{\href{https://www.vagrantup.com/}{Strona projektu Vagrant}} - tworzenie obrazów maszyn wirtualnych oraz ich uruchamianie
              \item libvirt\footnote{\href{https://libvirt.org/}{Strona projektu libvirt}} - zarządzanie maszynami wirtualnymi
              \item OpenLDAP\footnote{\href{https://www.openldap.org/}{Strona projektu OpenLDAP}} - dostępu do systemu katalogowego
              \item NFS\footnote{\href{https://docs.microsoft.com/en-us/windows-server/storage/nfs/nfs-overview}{Opis na stronie firmy Microsoft}} - dostęp do katalogów domowych z maszyny wirtualnej
              \item Arch Linux\footnote{\href{https://archlinux.org/}{Strona systemu operacyjnego Arch Linux}} - system operacyjny uruchamiany przez maszyny wirtualne
              \item GNU/Linux - wspierany system operacyjny
          \end{itemize}
    \item Różne
          \begin{itemize}
              \item Swagger Codegen\footnote{\href{https://swagger.io/tools/swagger-codegen/}{Opis narzędzia na stronie firmy Swagger}} - automatyczna generacja API na podstawie specyfikacji
              \item RDP\footnote{\href{https://docs.microsoft.com/en-us/troubleshoot/windows-server/remote/understanding-remote-desktop-protocol}{Dokumentacja protokołu RDP od Microsoft}} - łączenie ze zdalnymi sesjami
          \end{itemize}
\end{itemize}

\end{document}
