\documentclass[../deliverable-two.tex]{subfiles}

\begin{document}
\label{external-modules}

\subsection{Broker wiadomości}

Komunikacje wewnątrz systemu, czyli pomiędzy serwerami wirtualizacji oraz nadzorcami, będzie realizowali poprzez kolejki wiadomości.
Wiadomości będą przekazywane pomiędzy modułami przez brokera, którego nie będziemy implementować.
Zdecydowaliśmy się na skorzystanie z systemu RabbitMQ, który zapewni niezawodność komunikacji pomiędzy modułami.
Pozostaje jednak problem hazardu oraz wyścigów, które zostaną wyeliminowane w logice komunikacji nadzorcy oraz serwera wirtualizacji.

Aby system mógł funkcjonować zdefiniowane zostaną kolejki:
\begin{enumerate}[label=(\Roman*)]
	\item \label{virtsrv} Kolejka kończąca się na każdym z serwerów wirtualizacji i dla każdego z nich wiadomości sa powielane.
	Służy ona do wysyłania próśb od nadzorców do serwerów wirtualizacji.
	\item \label{overseers} Kolejka kończąca się na każdym z nadzorców i dla każdego z nich wiadomości są powielane.
	Służy ona do wysyłania informacji do nadzorców o zmianie stanu w serwerze wirtualizacji.
	Dodatkowo może służyć do wymiany danych pomiędzy nadzorcami.
	\item \label{exclusive} Kolejka kończąca się wyłącznie na pojedynczym serwerze wirtualizacji.
	Liczba kolejek zgadza się z liczbą serwerów wirtualizacji aktywnych w systemie.
	Służą one do sprawdzania, czy serwer wirtualizacji nadal pracuje po drugiej stronie.
	Skorzystamy z funkcjonalności kolejek na wyłączność(Exclusive Queue\footnote{\href{https://www.rabbitmq.com/queues.html\#exclusive-queues}{Opis zachowania kolejek na wyłączność}}).
	Każda z nich powinna mieć nazwę jaka przedstawi się serwer wirtualizacji.
\end{enumerate}

Powyższe 3 grupy kolejek umożliwią prawidłowe działanie systemu.
Każdy z modułów utworzy odpowiednie kolejki w trakcie uruchamiania.
Jedynym wymogiem prawidłowego uruchomienia komunikacji jest dostępny przez wszystkie serwery wirtualizacji oraz nadzorców proces brokera.

\subsection{Dysk sieciowy}
Usługa wspólnej przestrzeni dyskowej jest potrzebna do uzyskania niezależności wyboru maszyny wirtualnej od folderu użytkownika.
Każda maszyna wirtualna powinna przy starcie otrzymać adres oraz dane dostępowe do takiego dysku sieciowego.
Dane zostaną dostarczone poprzez wykonanie Ansible playbooka po uruchomieniu maszyny.

W przypadku maszyn wirtualnych uruchamiających system Linux potrzebne są dane do połączenia przez protokół NFS, a przy systemie Windows dane do protokołu SAMBA.
Niezależnie od protokołu dane nie mogą się różnić(struktura katalogów oraz zawartość plików).

\subsection{System katalogowy}
Usługa systemu katalogowego jest potrzebna do uzyskania niezależności wyboru maszyny wirtualnej od danych logowania do systemu uruchomionego na maszynie wirtualnej.
Każda maszyna wirtualna powinna przy starcie otrzymać adres oraz dane dostępowe do takiego systemu katalogowego
Dane zostaną dostarczone poprzez wykonanie Ansible playbooka po uruchomieniu maszyny.

Aby system katalogowy był kompatybilny z systemami Windows oraz Linux uruchamianymi na maszynie wirtualnej skorzystamy z protokołu OpenLDAP do uzyskiwania danych o użytkownikach.

\end{document}