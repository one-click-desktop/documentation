\documentclass[../../deliverable-two.tex]{subfiles}
\graphicspath{
  {\subfix{../../../api/}}
}

\begin{document}

\subsection{Komunikacja użytkownika z systemem - REST API}

Komunikacja aplikacji klienckiej oraz panelu administratora z systemem - nadzorcą - rozwiązana jest za pomocą REST API\footnote{\href{https://restfulapi.net/}{Opis REST API}}. Wiadomości wysyłane są za pomocą protokołu HTTPS\footnote{\href{https://datatracker.ietf.org/doc/html/rfc2818}{Specyfikacja protokołu HTTP Over TLS}}, który zapewnia ich szyfrowanie. W tym celu wymagane jest, aby na adres, pod którym udostępniony będzie system, wystawiony był odpowiedni certyfikat\footnote{\href{https://protonmail.com/blog/tls-ssl-certificate/}{Opis certyfikatu TLS/SSL}}, gwarantujący jego tożsamość. Podczas tworzenia systemu i testów możliwe jest użycie sztucznego, własnoręcznie podpisanego certyfikatu\footnote{\href{https://aboutssl.org/what-is-self-sign-certificate/}{Opis własnoręcznie podpisanego certyfikatu TLS/SSL}}.

Całość specyfikacji API umieszczona jest w osobnym pliku. Poniżej znajduje się zestawienie oraz krótki opis endpointów.

\begin{figure}[H]
    \centering\includegraphics[width=0.9\textwidth]{endpoints.png}
    \caption{Endpointy API}
\end{figure}

\begin{itemize}
    \item Login - służy do logowania do systemu; współdzielony przez aplikację kliencką oraz panel administracyjny. Poprawne zalogowanie zwraca token do dalszej autoryzacji.
    \item Machines - służy do pobierania przez aplikację informacji o typach i ilości dostępnych maszyn. Utworzenie sesji jest możliwe poprzez \texttt{POST} z typem maszyny. W odpowiedzi użytkownik dostaje częściowo wypełniony obiekt sesji zawierający id umożliwiające dalsze zapytania. \texttt{GET} zwraca obiekt sesji z aktualnym stanem. Jeżeli sesja jest gotowa, to zawiera on też adres, z którym należy nawiązać połączenie RDP. Ten endpoint, oraz wszystkie następne wymagają autoryzacji poprzez umieszczenie tokenu otrzymanego podczas logowania w odpowiednim nagłówku wiadomości, oraz dostępne są tylko dla użytkownika.
    \item Session - pozwala na wysłanie prośby o uzyskanie sesji, pobranie stanu sesji oraz jej anulowanie.
    \item Resources - udostępnia informację o zasobach działających serwerów wirtualizacji. Dostępny jedynie dla administratora.
\end{itemize}

\subsection{Komunikacja wewnątrz systemu - broker wiadomości}

Komunikacja wewnątrz systemu opiera się na kolejkach opisanych w \nameref{external-modules}. W celu uniknięcia wyścigów i utrzymania spójności modelu systemu pomiędzy nadzorcami ustalone są następujące zasady:
\begin{itemize}
    \item Nadzorca może zmienić stan systemu jedynie w reakcji na odpowiedź serwera wirtualizacji. Odpowiedzi te wysyłane są do wszystkich nadzorców, dzięki czemu każdy nadzorca ma taki sam model systemu.
    \item Wiadomości procesowane są przez serwer wirtualizacji w sposób atomowy. Pojedyncza wiadomość musi zostać w pełni obsłużona zanim program przejdzie do obsługi kolejnej.
    \item Serwer wirtualizacji odpowiada na wiadomości wysyłając nowy stan maszyn. Jeżeli żądanie nie może być spełnione z powodu błędnego żądania, to serwer nie odpowiada na żądanie. Wyjątkiem jest żądanie o wysłanie aktualnego stanu maszyn.
    \item Z powodu asynchroniczności wiadomości moduły nie oczekują na odpowiedź. W przypadku nadzorcy przetwarzanie "odpowiedzi" zostanie uruchomione przez zmianę modelu.
    \item Do monitorowania utrzymania połączenia z brokerem użyty jest wbudowany mechanizm, który umożliwia wywołanie odpowiedniej procedury, gdy moduł nie wyśle wiadomości o podtrzymaniu połączenia przez określony czas. Używając tego nadzorcy wykrywają, kiedy poszczególne serwery wirtualizacji przestaną działać, a serwery wirtualizacji - kiedy wszyscy nadzorcy przestaną działać.
\end{itemize}

Opisane wyżej założenia pozwalają uniknąć problemu hazardów i wyścigów. Jeżeli wiele nadzorców wyśle do serwera wirtualizacji tą samą prośbę, np. o stworzenie sesji na konkretnej maszynie, to z atomowości obsługi sesja zostanie stworzona tylko dla pierwszego z nich. Serwer wirtualizacji wyśle wiadomość o aktualizacji stanu maszyn i zignoruje pozostałe prośby. Nadzorcy otrzymają zmianę stanów, co spowoduje wywołanie odpowiednich procedur. Dla pierwszego będzie to dalsza część procesu tworzenia sesji, a pozostali nadzorcy pozostaną w procesie wyszukiwania maszyny do sesji.

\end{document}