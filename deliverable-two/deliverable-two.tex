\documentclass[12pt]{article}

\usepackage{polski}
%\usepackage[utf8]{inputenc}
%\usepackage[polish]{babel}
\usepackage[T1]{fontenc}
\usepackage{graphicx}
\usepackage{caption}
\usepackage{float}
\usepackage{footnote}
\usepackage{hyperref}
\usepackage{hyperref}
\usepackage{enumitem}

\graphicspath{
    {../screens}
    {../diagrams}
    {../api}
}

\def \screenswidth {0.6\textwidth}

\usepackage{subfiles}

\textheight 23.2 cm
\textwidth 6.0 in
\hoffset = -0.5 in
\voffset = -2.4 cm

\begin{document}

\section{BLa bla bla - wymagane punkty jak z deliverable-one (MODULARNOSC!!!)}

\section{Architektura systemu} % Krzysiu

\subfile{sections/architecture.tex}
\pagebreak

\section{Opis tworzonych modułów} % Piotrek

\subfile{sections/modules.tex}
\pagebreak

\section{Opis zewnętrznie dostarczonych modułów} % Krzysiu

\subfile{sections/external-modules.tex}
\pagebreak

\section{Komunikacja} % Piotrek

\subfile{sections/communication.tex}
\pagebreak

\section{Diagramy} % Krzysiu

\subfile{sections/diagrams.tex}
\pagebreak

\section{Interfejs użytkownika} % Piotrek

\subfile{sections/user-interface.tex}
\pagebreak

\section{Zewnętrzne narzędzia} % Krzysiu

\subfile{sections/external-tools.tex}
\pagebreak

\section{Wybrana technologia} % Piotrek

\subfile{sections/technology.tex}
\pagebreak

\end{document}
