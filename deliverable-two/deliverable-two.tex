\documentclass[12pt]{article}

\usepackage{polski}
%\usepackage[utf8]{inputenc}
%\usepackage[polish]{babel}
\usepackage[T1]{fontenc}
\usepackage{graphicx}
\usepackage{caption}
\usepackage{float}
\usepackage{footnote}
\usepackage{hyperref}
\usepackage{hyperref}
\usepackage{enumitem}
\usepackage{datetime}
\usepackage{xparse}

\newdateformat{mydate}{\twodigit{\THEDAY}.\twodigit{\THEMONTH}.\THEYEAR r.}

\ExplSyntaxOn
\bool_new:N \g_fifikinz_first_rev_bool
\cs_new:Nn \fifikinz_revision:nnn {
	% We only want to start a new row after an existing row (i.e., just
	% not the first row).  Once we pass the first row, set our flag to
	% `false' so it will insert \\ on subsequent calls.
	\bool_if:NTF \g_fifikinz_first_rev_bool {
		\bool_gset_false:N \g_fifikinz_first_rev_bool
	} { \\ }
	#1 & #2 & \ignorespaces#3\unskip
}

\NewDocumentEnvironment { Revisions } { } {
	% The first revision will be, well, the first revision.
	\bool_gset_true:N \g_fifikinz_first_rev_bool

	% make our table look a little nicer
	\renewcommand\arraystretch{1.5}

	\section*{Historia \space zmian}
	\begin{tabular}{crp{.8\linewidth}}
	} { \end{tabular} \bigskip }

\NewDocumentCommand \Revision { m m +m } {
	% Using \NewDocumentCommand allows some advanced syntax for
	% parameters.  See `texdoc xparse' for more details.
	\fifikinz_revision:nnn {#1} {#2} {#3}
}
\ExplSyntaxOff

\graphicspath{
    {../screens}
    {../diagrams}
    {../api}
}

\def \screenswidth {0.6\textwidth}
\def \screenswidthlesser {0.5\textwidth}

\usepackage{subfiles}

\textheight 23.2 cm
\textwidth 6.0 in
\hoffset = -0.5 in
\voffset = -2.4 cm

\begin{document}

\subfile{sections/intro.tex}
\pagebreak

\section{Architektura systemu} % Krzysiu

\subfile{sections/architecture.tex}
\pagebreak

\section{Opis tworzonych modułów} % Piotrek

\subfile{sections/modules.tex}
\pagebreak

\section{Opis zewnętrznie dostarczonych modułów} % Krzysiu

\subfile{sections/external-modules.tex}
\pagebreak

\section{Komunikacja} % Piotrek

\subfile{sections/communication.tex}
\pagebreak

\section{Diagramy} % Krzysiu

\subfile{sections/diagrams.tex}
\pagebreak

\section{Interfejs użytkownika} % Piotrek

\subfile{sections/user-interface.tex}
\pagebreak

\section{Zewnętrzne narzędzia} % Krzysiu

\subfile{sections/external-tools.tex}
\pagebreak

\section{Wybrana technologia} % Piotrek

\subfile{sections/technology.tex}
\pagebreak

\section{Załączniki}

\subfile{sections/attachments.tex}

\end{document}
