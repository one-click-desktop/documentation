\documentclass[../praca-dyplomowa.tex]{subfiles}

\begin{document}

\begin{titlepage}
    \begin{flushright}
        {\mydate\today}\\
    \end{flushright}
    \vskip30ex

    \begin{center}
        \Large {\bf{
                System do zdalnej pracy w środowisku graficznym wykorzystujący maszyny wirtualne QEMU z akceleracja sprzętową\\
            }}
        \vskip2ex
        \bf{Architektura i opis systemu\\}
        \vskip2ex
        \small { Autorzy: Krzysztof Smogór, Piotr Widomski\\  }
        \small { Promotor: Dr inż. Marek Kozłowski\\ }
        \small { Wersja 1.2 }
    \end{center}
\end{titlepage}

\abstract
Dokument ma za zadanie zapoznać czytelnika z szczegółowym opisem systemu.
Wpierw opisane są tworzone oraz zewnętrznie dostarczone moduły.
Następnie opisane zostały metody komunikacji między modułami.
Dalej, za pomocą diagramów, opisane zostały zależności i współdziałanie modułów.
Przedstawione zostały modele interfejsu użytkownika dla aplikacji klienckiej oraz panelu administratora.
Kolejno opisane zostały używane narzędzia oraz technologie.
Na końcu dokumentu znajduje się lista załączników.

\begin{Revisions}
    \Revision{1.0}{05.11.2021r.}{
        Pierwsza wersja.
    }
    \Revision{1.1}{06.11.2021r.}{
        Opis modułów, komunikacja, interfejs użytkownika oraz technologie
    }
    \Revision{1.2}{07.11.2021r..}{
        Diagramy, zewnętrzne narzędzia oraz wstęp
    }
\end{Revisions}

\tableofcontents

\end{document}