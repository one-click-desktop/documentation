\documentclass[../opis-rozwiazania.tex]{subfiles}

\begin{document}
\label{system_startup}

W~tym podrozdziale zakładamy, że każdy system operacyjny został skonfigurowany poprawnie zgodnie z~opisem w~rozdziale \ref{system_requirements}.

\subsection{Pozyskanie aplikacji klienckiej}
\label{system_startup.client_obtaining}
Zalecanym sposobem pozyskania aplikacji klienckiej jest udanie się do sekcji z~oficjalnymi wydaniami \parencite{ocd-client-releases} modułu aplikacji klienckiej oraz pobranie najnowszej wersji dla wybranego systemu operacyjnego.
Przy pierwszym uruchomieniu trzeba pobrać archiwum plików zawierające aplikację oraz plik konfiguracyjny.
Bez pliku konfiguracyjnego aplikacja nie uruchomi się.

\subsection{Budowanie kontenerów}
\label{system_startup.containers}

Moduły: panelu administracyjnego, nadzorcy i~serwera wirtualizacji można uruchomić pod postacią kontenera dockerowego.
Jest to zalecany sposób uruchamiania tych trzech modułów.

\subsubsection{Ujednolicony sposób budowania kontenerów}
Każdy z~tych 3 modułów ma przygotowany skrypt \texttt{build.sh}, który powinien prawidłowo zbudować kontener.
Skrypt ten wykona polecenie budowania i~oznaczy kontener odpowiednią nazwą.
Należy wykonać go zawsze z~poziomu głównego folderu repozytorium modułu.
Każdy kontener przy starcie wykona skrypt \texttt{assets/entry\_point.sh}.

\subsubsection{Budowanie kontenera serwera wirtualizacji}
Kontener serwera wirtualizacji wymaga specjalnie przygotowanego wcześniej kontenera zawierającego wszystkie potrzebne zależności do uruchomienia aplikacji.
Aby go zbudować należy skorzystać z~pliku \texttt{\seqsplit{runtime\_container/Dockerfile}} i~oznaczyć go nazwą \texttt{\seqsplit{click-desktop/virtualization-server-runtime}}.
Kontener zawierający aplikację w~czasie budowania będzie oczekiwał, że taki obraz istnieje.

Po zbudowaniu kontenera z~zależnościami można przystąpić do budowania kontenera głównego.
Należy do tego celu skorzystać z~dostarczonego pliku \texttt{Dockerfile} w~głównym folderze repozytorium.

Przy uruchomieniu skryptu \texttt{build.sh} kontener jest budowany z~nazwą \texttt{one-click-desktop/virtualization-server}.

\subsubsection{Budowanie kontenera nadzorcy i~panelu administratora}
W~przypadku tych dwóch modułów nie trzeba wykonać żądnych dodatkowych przygotowań.
Wystarczy skorzystać z~dostarczonego pliku \texttt{Dockerfile} w~głównym folderze repozytorium.

Przy uruchomieniu skryptu \texttt{build.sh} kontenery są budowane z~nazwami odpowiednio \texttt{one-click-desktop/overseer} oraz \texttt{\seqsplit{one-click-desktop/admin-panel}}.

\subsection{Budowanie modułów}
Każdy z~modułów posiada dokładną instrukcję budowania w~pliku \texttt{README.md}.
Uruchamianie zbudowanych modułów zaleca się tylko w~przypadku rozwoju aplikacji.
Przy wdrażaniu systemu najszybciej i~najbezpieczniej jest skorzystać z~metod opisanych w~sekcjach \ref{system_startup.client_obtaining} i~\ref{system_startup.containers}.

\subsubsection{Moduły napisane w~technologii .NET}
W~przypadku budowania modułu nadzorcy i~serwera wirtualizacji potrzebne są narzędzia deweloperskie .NET 5.0.
W~trakcie budowania aplikacji wszystkie użyte biblioteki zostaną pobrane przy użyciu menadżera pakietów Nuget.
Przy uruchomieniu serwera wirtualizacji potrzebny jest, poza wymaganiami zdefiniowanymi w~rozdziale \ref{system_requirements.virtsrv_rquirements}, zainstalowany Vagrant wraz z~wtyczka vagrant-libvirt w~wersji przynajmniej 0.7.1 (wersje wcześniejsze zawierają błąd uniemożliwiający uruchomienie maszyny wirtualnej w~trybie UEFI \parencite{vagrant-libvirt-nvram-error}).

\subsubsection{Moduły napisane w~technologii Node}
W~przypadku budowania modułów aplikacji klienckiej oraz panelu administracyjnego potrzebny będzie pakiet Node. Przed zbudowaniem aplikacji należy pobrać wszystkie wymagane biblioteki przy użyciu menadżera pakietów npm poleceniem \texttt{npm install}.

\subsection{Konfiguracja aplikacji klienckiej}
Aplikacja kliencka wymaga pliku konfiguracyjnego o~nazwie \texttt{config.json} w~tym samym folderze co plik wykonywalny.
W~pliku konfiguracyjnym użytkownik musi podać:
\begin{itemize}
  \item Adres jednego z~nadzorców lub serwera dostępowego (\texttt{basePath}) - adres musi być w~formacie URI.
  \item Adres zewnętrznego brokera wiadomości (\texttt{rabbitPath}) - adres musi być w~formacie URI.
  \item Czy aplikacja powinna podczas połączenia RDP używać danych dostępowych takich samych jak przy logowaniu do systemu (\texttt{useRdpCredentials}) - \texttt{true} albo \texttt{false}.
  \item Czy aplikacja powinna uruchamiać zintegrowanego klienta RDP (\texttt{startRdp}) - \texttt{true} albo \texttt{false}. Przydaje się to przy wykorzystaniu własnego klienta RDP. Aplikacja po uzyskaniu sesji wyświetli dane dostępowe do przypisanej maszyny wirtualnej.
\end{itemize}

\subsection{Konfiguracja panelu administracyjnego}
\label{system_startup.admin_panel_conf}
Przy uruchomieniu panelu administracyjnego trzeba przekazać adres jednego z~nadzorców lub serwera dostępowego (jeżeli taki jest używany).
Aby tego dokonać należy ustawić zmienną środowiskową \texttt{API\_URL} na adres dostępowy jednego z~nadzorców lub serwera dostępowego.

\subsection{Konfiguracja nadzorcy}
\label{system_startup.overseer_conf}
Nadzorca posiada wiele parametrów związanych z~działaniem całego systemu.
Kontroluje on między innymi czas oczekiwania na powrót użytkownika do porzuconej sesji.
Wszystkie parametry należy przekazać poprzez plik konfiguracyjny.

Nazwa pliku konfiguracyjnego musi spełniać wzorzec \texttt{\seqsplit{appsettings.\$\{Nazwa\_srodowiska\}.ini}}.
W~przypadku uruchamiania aplikacji wewnątrz kontenera ustawione jest środowisko o~nazwie \texttt{Production}.
Aplikacja nadzorcy domyślnie przeszukuje folder \texttt{./config} w~poszukiwaniu plików konfiguracyjnych.
Można zmienić ścieżkę do folderu konfiguracyjnego poprzez parametr \texttt{-c path/to/other/config/folder/}.

W~pliku konfiguracyjnym nadzorca odczytuje następujące sekcji:
\begin{itemize}
  \item \texttt{JwtSettings} - przechowuje parametr wykorzystywany do generowania unikatowych tokenów autoryzacyjnych
  \item \texttt{OneClickDesktop} - przechowuje parametry związane z~działaniem systemu OneClickDesktop.
  \item \texttt{LDAP} - przechowuje parametry związane z~integracją z~systemem katalogowym przechowującym użytkowników.
\end{itemize}
Poza tymi sekcjami można tam umieścić dowolne sekcje, które będzie przetwarzać ASP.NET \parencite{asp-conf}.
Jednak warte uwagi są następujące sekcje:
\begin{itemize}
  \item \texttt{Logging} - parametry loggera dostarczonego przez Microsoft (\url{https://www.nuget.org/packages/microsoft.extensions.logging}).
  \item \texttt{Kestrel} - parametry serwera HTTP udostępniającego aplikację. Aplikacjia nie zadziała w~trybie HTTPS bez skonfigurowanego certyfikatu.
\end{itemize}
\noindent
Pozostają jeszcze 2 ważne parametry bez sekcji zaimplementowane przez ASP.NET:
\begin{itemize}
  \item \texttt{AllowedHosts} - lista adresów z~których zapytania będą obsługiwane.
  \item \texttt{urls} - lista adresów, na których aplikacja będzie nasłuchiwać zapytań.
\end{itemize}

\subsubsection{Parametry sekcji \texttt{JwtSettings}}
Sekcja ta zawiera jedynie parametr \texttt{Secret}. Należy go ustawić na dowolny napis. W~przypadku uruchamiania więcej niż jednej instancji nadzorcy użycie identycznej wartości parametru może być pożądane. Pozwala to na bezproblemową zmianę nadzorcy, z~którym przebiega komunikacja, bez potrzeby ponownej autoryzacji.

\subsubsection{Parametry sekcji \texttt{OneClickDesktop}}
Sekcja zawiera parametry związane z~komunikacją pomiędzy jednostkami systemu oraz kilka parametrów określających interwały czasowe zdarzeń.
W~dostarczonej przykładowej konfiguracji wykomentowane wartości oznaczają wartości domyślne.
\begin{itemize}
  \item \texttt{OverseerId} - identyfikator nadzorcy używany w~komunikacji poprzez wewnętrznego brokera wiadomości. Powinna być unikatowa w~skali jednej instancji systemu OneClickDesktop. Domyślnie przyjmuje wartość \texttt{overseer-test}.
  \item \texttt{RabbitMQHostname} - adres dostępowy wewnętrznego brokera wiadomości. Domyślnie przyjmuje wartość \texttt{localhost}. Bez działającego procesu brokera pod tym adresem aplikacja wyłączy się z~błędem zaraz po uruchomieniu.
  \item \texttt{RabbitMQPort} - port dostępowy wewnętrznego brokera wiadomości. Domyślnie przyjmuje wartość \texttt{5672}. Bez działającego procesu brokera pod tym portem aplikacja wyłączy się z~błędem zaraz po uruchomieniu.
  \item \texttt{ModelUpdateInterval} - ilość sekund co ile nadzorca prosi serwery wirtualizacji o~aktualizację modelu. Domyślnie przyjmuje wartość \texttt{60}. Zalecane jest aby wartość tego parametru opisywała czas od 30 do 60 sekund.																												%wtf? ą nie działa we wnętrzu {}
  \item \texttt{DomainShutdownTimeout} - ilość minut jaka musi upłynąć od oznaczenia maszyny wirtualnej stanem \textit{Oczekiwanie na wyłączenie maszyny} do jej wyłączenia. Domyślnie przyjmuje wartość \texttt{15}.
  \item \texttt{DomainShutdownCounterInterval} - podany w~sekundach okres czasu pomiędzy sprawdzeniem czy maszyna wirtualna nadal oczekuje na zamknięcie. Zaleca się, aby ten czas dzielił czas z~parametru \texttt{DomainShutdownTimeout} na równe części. Takich części nie powinno być mniej niż 10, ale także nie więcej niż 100. Każde takie sprawdzenie zużywa czas procesora.
\end{itemize}

\subsubsection{Parametry sekcji \texttt{LDAP}}
Sekcja zawiera parametry związane z~integracją aplikacji z~istniejącym systemem katalogowym przechowującym użytkowników. Zalecamy użycie systemu wykorzystującego implementację OpenLDAP (\url{https://www.openldap.org/}). Integracja z~innymi rozwiązaniami może nie działać prawidłowo. Dodatkowo przechowywanie folderów użytkowników w~katalogu \texttt{\\home} może powodować konflikt z~lokalnymi użytkownikami znajdującymi się w~Vagrant Boxie.

Użytkownicy zapisani w~systemie katalogowym powinni dziedziczyć po klasie \texttt{posixAccount}. W~takim przypadku parametry związane z~nazwami atrybutów mogą przyjmować domyślną wartość.
\begin{itemize}
  \item \texttt{LdapHostname} - adres dostępowy systemu katalogowego. Domyślnie przyjmuje wartość \texttt{localhost}. Bez działającego systemu katalogowego pod tym adresem autoryzacja użytkowników nie będzie możliwa.
  \item \texttt{LdapPort} - port dostępowy systemu katalogowego. Domyślnie przyjmuje wartość \texttt{localhost}. Bez działającego systemu katalogowego pod tym portem autoryzacja użytkowników nie będzie możliwa.
  \item \texttt{Domain} - nazwa domeny systemu katalogowego. Jest to identyfikator korzenia drzewa katalogów. Domyślna wartość wynosi \texttt{dc=example,dc=org}.
  \item \texttt{UserSearchBase} - lokalizacja w~drzewie katalogów, pod którą znajdują się użytkownicy systemu OneClickDesktop. Lokalizacja musi być w~formie ścieżki bezwzględnej od korzenia. Domyślna wartość wynosi \texttt{ou=users,dc=example,dc=org}.
  \item \texttt{UseSsl} - informacja czy komunikacja z~systemem katalogowym powinna być wykonywana z~użyciem szyfrowanego protokołu. Wymaga poprawnego skonfigurowanych certyfikatów szyfrowania. Domyślna wartość wynosi \texttt{false}.
  \item \texttt{UserGuidAttribute} - Nazwa atrybutu przechowującego unikatowy w~skali systemu identyfikator użytkownika (GUID). Atrybut ten nie należy do domyślnych atrybutów znajdujących się w~LDAPie. Można do jego przechowywania wykorzystać pole przechowujące napis lub specjalny typ. Domyślna wartość wynosi \texttt{guid}.
  \item \texttt{UserGroupAttribute} - Nazwa atrybutu przechowującego identyfikator grupy, do której należy użytkownik. Atrybut musi zawierać wyłącznie jeden numer, który odpowiada grupie użytkowników lub administratorów systemu OneClickDesktop. Domyślna wartość wynosi \texttt{gidNumber}.
  \item \texttt{UserNameAttribute} - Nazwa atrybutu przechowującego nazwę użytkownika. Domyślna wartość wynosi \texttt{uid}.
  \item \texttt{UserGroupNumber} - identyfikator grupy użytkowników systemu OneClickDesktop. Domyślna wartość wynosi \texttt{502}.
  \item \texttt{AdminGroupNumber} - identyfikator grupy administratorów systemu OneClickDesktop. Domyślna wartość wynosi \texttt{502}.
  \item \texttt{ReadonlyUserName} - nazwa konta z~prawami do odczytu. Wykorzystywane jest do pobrania z~systemu katalogowego informacji o~konkretnym użytkowniku. Uprawnienia muszą pozwalać na odczyt użytkowników systemu OneClickDestop i~wymienionych wcześniej atrybutów. Domyślna wartość wynosi \texttt{readonly}.
  \item \texttt{ReadonlyUserPassword} - hasło dostępowe konta z~prawami odczytu. Domyślna wartość wynosi \texttt{readonly}.
\end{itemize}

\subsubsection{Przykładowa konfiguracja}

Konfiguracja ta umożliwia użycie protokołu HTTPS dzięki przekazaniu zabezpieczonego hasłem certyfikatu oraz nasłuchiwaniu na adresie obsługującym ten protokół.

\begin{verbatim}
	AllowedHosts=*
	urls=http://*:5000;https://*:5001

	[Kestrel:Certificates:Default]
	Password=password
	Path=/overseer/overseer.pfx

	[JwtSettings]
	Secret=MySecretForJWTTokensPleaseChangeMe

	[Logging.LogLevel]
	Default=Information
	Microsoft=Warning
	Microsoft.Hosting.Lifetime=Information

	[OneClickDesktop]
	OverseerId=overseer-instance-123
	RabbitMQHostname=1.2.3.4
	RabbitMQPort=5678
	ModelUpdateInterval=60
	DomainShutdownTimeout=15
	DomainShutdownCounterInterval=30

	[LDAP]
	LdapHostname=localhost
	LdapPort=389
	UserSearchBase="ou=users,dc=example,dc=org"
	Domain="dc=example,dc=org"
	UseSsl=false
	UserGuidAttribute=guid
	UserGroupAttribute=gidNumber
	UserNameAttribute=uid
	UserGroupNumber=502
	AdminGroupNumber=501
	ReadOnlyUserName=readonly
	ReadOnlyUserPassword=readonly
\end{verbatim}

\subsection{Konfiguracja serwera wirtualizacji}
\label{system_startup.virtsrv_conf}
Serwer wirtualizacji posiada wiele parametrów związanych z~zasobami przekazanymi do dyspozycji systemu.
Wszystkie parametry należy przekazać poprzez pliki konfiguracyjne znajdujące się w~jednym folderze.

Głównym plikiem konfiguracyjnym jest \texttt{virtsrv.ini}.
Zawiera on parametry związane z~komunikacją wewnątrz i~na zewnątrz systemu.
Dodatkowo znajdują się tam informacje o~zasobach udostępnionych dla serwera wirtualizacji.
Dodatkowe pliki konfiguracyjne reprezentują typy maszyn wirtualnych uruchamianych na tym serwerze wirtualizacji.
Jedynym odstępstwem od reguły jest plik konfiguracyjny dla pakietu NLog (\url{https://nlog-project.org/}).
Musi się on znajdować w~tym samym folderze co aplikacja oraz nazywać \texttt{NLog.config}.

Aplikacja serwera wirtualizacji domyślnie przeszukuje folder \texttt{./config} w~poszukiwaniu plików konfiguracyjnych.
Można zmienić ścieżkę do folderu konfiguracyjnego poprzez parametr \texttt{-c\ path/to/other/config/folder/}.

\subsubsection{Główny plik konfiguracyjny}
W~głównym pliku konfiguracyjnym możemy wyróżnić następujące sekcje:
\begin{itemize}
  \item \texttt{OneClickDesktop} - przechowuje parametry związane z~działaniem systemu OneClickDesktop.
  \item \texttt{Ldap} - zawiera parametry systemu katalogowego przechowującego użytkowników.
  \item \texttt{Nfs} - przechowuje dane dysku sieciowego, na którym znajdują się katalogi domowe użytkowników.
  \item \texttt{ServerResources} - przechowuje parametry określające wszystkie udostępnione zasoby.
\end{itemize}

W~sekcji \texttt{OneClickDesktop} występują parametry:
\begin{itemize}
  \item \texttt{VirtualizationServerId} - identyfikator serwera wirtualizacji używany w~komunikacji poprzez wewnętrznego brokera wiadomości. Powinna być unikatowa w~skali jednej instancji systemu OneClickDesktop. Domyślnie przyjmuje wartość \texttt{virtsrv-test}.
  \item \texttt{OversserCommunicationShutdownTimeout} - po upływie tylu sekund bez komunikacji od jakiegokolwiek nadzorcy serwer wirtualizacji uznaje, że zabrakło nadzorców w~systemie. Nastąpi wtedy wyłączenie aplikacji serwera wirtualizacji. Domyślnie parametr przyjmuje wartość \texttt{120}. Zaleca się, aby wartość parametru była większa niż dwukrotność parametru \texttt{ModelUpdateInterval} z~konfiguracji nadzorcy opisanej w~sekcji \ref{system_startup.overseer_conf}.
  \item \texttt{LibvirtUri} - adres do połączenia z~usługą libvirta. Vagrant i~biblioteka libvirta skorzystają z~tego adresu do nawiązania łączności. Specyfikacja libvirta dokładnie opisuje format tego adresu \parencite{libvirt-uri}.
  \item \texttt{VagrantFilePath} - ścieżka do specjalnie przygotowanego pliku wsadowego dla Vagranta. Wykorzystywany jest przez system do uruchamiania i~wyłączania maszyn wirtualnych. Aplikacja nie zadziała prawidłowo bez ustawionej wartości parametru.
  \item \texttt{PostStartupPlaybook} - skrypt wykonywany przy pomocy Ansible'a zaraz po uruchomieniu maszyny wirtualnej. Koniecznie należy przetestować, czy konfiguracja wykonuje się prawidłowo. Przy każdym błędzie wykonania skryptu maszyna wirtualna zostanie usunięta. W~przypadku ustawienia pustej wartości Vagrant nie wykona operacji zaopatrzenia maszyny. Domyślnie parametr przyjmuje pustą wartość.
  \item \texttt{VagrantboxUri} - identyfikator szablonowej maszyny wirtualnej. Musi ona spełniać szereg wymogów opisanych w~sekcji \ref{system_requirements.vagrant_box}. W~przeciwnym wypadku system nie będzie w~stanie uruchomić takiego szablonu. Aplikacja nie zadziała prawidłowo bez ustawionej wartości parametru.
  \item \texttt{UefiPath} - ścieżka do obrazu UEFI znajdującego się na maszynie uruchamiającej. Prawidłowo ustawiona wartość jest wymagana do poprawnego zbudowania Vagrant Boxa. Gdy wartość nie zostanie ustawiona maszyna włączy się w~trybie BIOS.
  \item \texttt{NvramPath} - ścieżka do obrazu NVRAM wymaganego do prawidłowego uruchomienia maszyny w~trybie UEFI. Przed uruchomieniem maszyny wykonywana jest kopia do \texttt{/var/lib/libvirt/qemu/nvram/}, w~celu uniknięcia usunięcia oryginalnego obrazu podczas jej wyłączania. Gdy wartość nie zostanie ustawiona maszyna włączy się w~trybie BIOS. W~przypadku uruchamiania przy użyciu Dockera należy podać ścieżkę wewnątrz kontenera. Kopiowanie musi odbyć się do wolumenu, którego zawartość jest dostępna z~poziomu systemu uruchamiającego.
  \item \texttt{BridgeInterfaceName} - nazwa interfejsu sieciowego w~skonfigurowanego w~trybie bridge, do którego zostanie podłączona każda uruchomiona maszyna wirtualna. Domyślnie przyjmuje wartość \texttt{br0}.
  \item \texttt{BridgedNetwork} - adres sieci, do której podłączona zostanie maszyna wirtualna poprzez \texttt{BridgeInterfaceName}, podany w~formacie CIDR. Jeżeli maszyna nie będzie posiadać adresu z~podanej sieci po uruchomieniu, to zostanie wyłączona. Aplikacja nie zadziała prawidłowo bez ustawionej wartości parametru.
  \item \texttt{InternalRabbitMQHostname} - adres dostępowy wewnętrznego brokera wiadomości. Domyślnie przyjmuje wartość \texttt{localhost}. Bez działającego procesu brokera pod tym adresem aplikacja wyłączy się z~błędem zaraz po uruchomieniu.
  \item \texttt{InternalRabbitMQPort} - port dostępowy wewnętrznego brokera wiadomości. Domyślnie przyjmuje wartość \texttt{5672}. Bez działającego procesu brokera pod tym portem aplikacja wyłączy się z~błędem zaraz po uruchomieniu.
  \item \texttt{ExternalRabbitMQHostname} - adres dostępowy zewnętrznego brokera wiadomości. Domyślnie przyjmuje wartość \texttt{localhost}. Bez działającego procesu brokera pod tym adresem aplikacja wyłączy się z~błędem zaraz po uruchomieniu.
  \item \texttt{ExternalRabbitMQPort} -  port dostępowy zewnętrznego brokera wiadomości. Domyślnie przyjmuje wartość \texttt{5673}. Bez działającego procesu brokera pod tym portem aplikacja wyłączy się z~błędem zaraz po uruchomieniu.
  \item \texttt{ClientHeartbeatChecksForMissing} - liczba nieudanych sprawdzeń czy aplikacja kliencka nadal jest połączona do maszyny wirtualnej. Po tej liczbie prób maszyna wirtualna oznaczana jest jako oczekująca na zamknięcie. Domyslnie przyjmuje wartośc \texttt{2}.
  \item \texttt{ClientHeartbeatChecksDelay} - czas w~milisekundach pomiędzy sprawdzeniami, czy aplikacja kliencka nadal jest połączona do maszyny wirtualnej. Domyślnie przyjmuje wartość \texttt{10000}
\end{itemize}

Sekcja \texttt{Ldap} zawiera parametry przekazywane do tworzonej maszyny wirtualnej w~celu umożliwienia autoryzacji za pośrednictwem systemu katalogowego. Dostępne parametry to:
\begin{itemize}
	\item \texttt{Uri} - adres systemu katalogowego w~formacie URI. Bez działającego systemu katalogowego pod tym adresem autoryzacja użytkowników nie będzie możliwa. Domyślnie przyjmuje wartość \texttt{ldpa://localhost}. Domyślnie przyjmuje wartość \texttt{ldpa://localhost}.
	\item \texttt{Domain} - nazwa domeny systemu katalogowego. Jest to identyfikator korzenia drzewa katalogów. Domyślna wartość wynosi \texttt{dc=example,dc=org}.
	\item \texttt{ReadonlyUserName} - lokalizacja w~drzewie katalogów, pod którą znajduje się konto z~prawami do odczytu. Jeżeli system pozwala na anonimowy odczyt, to możliwe jest przekazanie pustej wartości. Domyślnie przyjmuje wartość \texttt{cn=readonly,dc=example,dc=org}.
	\item \texttt{ReadonlyUserPassword} - hasło dostępowe konta z~prawami odczytu. Domyślna wartość wynosi \texttt{readonly}.
	\item \texttt{AdminDn} - lokalizacja w~drzewie katalogów, pod którą znajduje się konto administracyjne. Używane do poprawnego wykonywania zmian w~systemie katalogowym. Domyślnie przyjmuje wartość \texttt{cn=admin,dc=example,dc=org}.
	\item \texttt{GroupsDn} - lokalizacja w~drzewie katalogów, pod którą znajdują się grupy użytkowników systemu OneClickDesktop. Domyślnie przyjmuje wartość \texttt{ou=groups,dc=example,dc=org}.
	\item \texttt{UsersDn} - lokalizacja w~drzewie katalogów, pod którą znajdują się użytkownicy systemu OneClickDesktop. Domyślnie przyjmuje wartość \texttt{ou=users,dc=example,dc=org}.
\end{itemize}

Sekcja \texttt{Nfs} zawiera następujące parametry związane z~używaniem dysku sieciowego. Katalogi użytkowników montowane są wewnątrz maszyny wirtualnej za pomocą sieciowego systemu plików NFS \parencite{nfs}. Z~tego powodu dysk sieciowy musi udostępniać katalogi użytkownika przy użyciu serwera NFS. Powiązane parametry to:
\begin{itemize}
	\item \texttt{ServerName} - adres dostępowy dysku sieciowego. Bez działającego, lub osiągalnego z~poziomu maszyny wirtualnej, serwera NFS pod tym adresem przenoszenie danych między sesjami nie będzie możliwe. Domyślna wartość wynosi \texttt{localhost}.
	\item \texttt{HomePath} - ścieżka wewnątrz serwera do katalogu przechowującego foldery użytkowników. Domyślna wartość wynosi \texttt{/}.
\end{itemize}

W~sekcji \texttt{ServerResources} występują parametry:
\begin{itemize}
  \item \texttt{Cpus} - liczba wątków, które może wykorzystać system. Domyślnie przyjmuje wartość \texttt{2}.
  \item \texttt{Memory} - ilość pamięci operacyjnej w~MiB jaką może wykorzystać system. Domyślnie przyjmuje wartość \texttt{2048}.
  \item \texttt{Storage} - ilość przestrzeni dyskowej w~GiB jaką może wykorzystać system. Domyślnie przyjmuje wartość \texttt{100}.
  \item \texttt{GPUsCount} - ilość kart graficznych przekazanych do systemu. Domyślnie przyjmuje wartość \texttt{0}.
  \item \texttt{MachineTypes} - lista typów maszyn wirtualnych. Każda nazwa typu jest oddzielona od sąsiednich przecinkiem. Nazwa typu musi składać się ze znaków opisanych następującym wyrażeniem regularnym \texttt{[a-zA-Z0-9\textbackslash-]}. Dla każdej z~nazw wyszukiwany jest plik konfiguracyjny o~nazwie zaczynającej się nazwą typu i~kończącą się \texttt{\_template.ini}.
\end{itemize}

Gdy $n = \texttt{GPUsCount}$ jest większy od 0, wtedy w~pliku muszą znaleźć się sekcje od \texttt{ServerGPU.1} do \texttt{ServerGPU.$n$} włącznie.
W~przeciwnym wypadku serwer wirtualizacji zakończy się z~błędem zaraz po uruchomieniu.

Każda z~tych sekcji zawiera parametr \texttt{AddressCount}, który określa jak duża jest grupa IOMMU w~której znajduje się przekazywane urządzenie PCI.
W~przypadku $m = \texttt{AddressCount}$ w~tej sekcji muszą znaleźć się parametry od \texttt{Address\_1} do \texttt{Address\_$m$} włącznie.
W~przeciwnym wypadku serwer wirtualizacji zakończy się z~błędem zaraz po uruchomieniu.
Każdy z~tych parametrów opisuje adres pojedynczego urządzenia na magistrali PCI.
Adres na magistrali PCI musi zostać opisany w~formacie uzyskanym z~polecenia \texttt{lspci} - \texttt{\$\{domain:4\}:\$\{bus:2\}:\$\{slot:2\}.\$\{function:1\}}.

Przykładowa zawartość głównego pliku konfiguracyjnego:
\begin{verbatim}
	[OneClickDesktop]
	VirtualizationServerId=virtsrv-instance-123
	OversserCommunicationShutdownTimeout=120
	LibvirtUri=qemu:///system
	VagrantFilePath=res/Vagrantfile
	VagrantboxUri=smogork/archlinux-rdp
	UefiPath=/usr/share/edk2-ovmf/x64/OVMF_CODE.fd
	NvramPath=/app/nvram.fd
	PostStartupPlaybook=res/poststartup_playbook.yml

	BridgeInterfaceName=br0
	BridgedNetwork=1.2.3.0/24

	InternalRabbitMQHostname=1.2.3.4
	InternalRabbitMQPort=5678

	ExternalRabbitMQHostname=1.2.3.4
	ExternalRabbitMQPort=8765

	ClientHeartbeatChecksForMissing=2
	ClientHeartbeatChecksDelay=10000

	[Ldap]
	Uri=ldap://1.2.3.4
	Domain=dc=example,dc=org
	ReadOnlyDn=cn=readonly,dc=example,dc=org
	ReadOnlyPassword=readonly
	AdminDn=cn=admin,dc=example,dc=org
	GroupsDn=ou=groups,dc=example,dc=org
	UsersDn=ou=users,dc=example,dc=org

	[Nfs]
	ServerName=1.2.3.4
	HomePath=/

	[ServerResources]
	Cpus=6
	Memory=4096
	Storage=200
	GPUsCount=1
	MachineTypes=cpu,cpu-memory

	[ServerGPU.1]
	AddressCount=2
	Address_1=0000:03:00.0
	Address_2=0000:03:00.1
\end{verbatim}

\subsubsection{Pliki konfiguracyjne opisujące typy maszyn wirtualnych}
Plik opisuje wymagane zasoby do uruchomienia zdefiniowanego typu maszyny wirtualnej.

Każdy z~typów jest opisany oddzielnym plikiem oraz posiada swoją nazwę.
Oznaczmy ją jako \texttt{\$\{template\_name\}}.
Nazwa typu powinna być unikatowa w~skali systemu.
Jeżeli chcemy udostępnić ten sam typ na wielu serwerach wirtualizacji, należy nadać mu takie same zasoby.
System może nieprawidłowo zarządzać zasobami, gdy uzyska dwa różne wzorce zasobów dla tego samego typu maszyny.
Plik opisujący typ musi mieć nazwę \texttt{\$\{template\_name\}\_template.ini} oraz znajdować się w~folderze z~pozostałymi plikami konfiguracyjnymi.
Jeżeli dla jakiejkolwiek nazwy typu aplikacja nie znajdzie pliku konfiguracyjnego, to zakończy się z~błędem zaraz po uruchomieniu.

W~znalezionym pliku musi być sekcja o~nazwie \texttt{\$\{template\_name\}\_template} z~parametrami:
\begin{itemize}
  \item \texttt{HumanReadableName} - nazwa wyświetlana w~aplikacji klienckiej. Domyślnie przyjmuje wartość \texttt{\$\{template\_name\}}.
  \item \texttt{Cpus} - liczba wątków potrzebna do uruchomienia maszyny. Domyślnie parametr przyjmuje wartość \texttt{2}.
  \item \texttt{Memory} - ilość pamięci operacyjnej w~MiB potrzebnej do uruchomienia maszyny. Domyślnie parametr przyjmuje wartość \texttt{512}.
  \item \texttt{Storage} - ilość przestrzeni dyskowej w~GiB potrzebnej do uruchomienia maszyny. Domyślnie parametr przyjmuje wartość \texttt{20}.
  \item \texttt{AttachGpu} - informacja czy maszyna powinna mieć przekazaną kartę graficzną. Domyślnie parametr przyjmuje wartość \texttt{false}.
\end{itemize}

Przykładowa zawartość pliku konfiguracyjnego typu maszyny:

\begin{verbatim}
	[gpu_template]
	HumanReadableName=Efficiency machine with GPU attached
	Cpus=1
	Memory=2048
	Storage=55

	AttachGpu = true
\end{verbatim}

\subsection{Procedura uruchomienia}
\label{system_startup.procedure}
Prawidłowy start systemu powinien być wykonany w~następującej kolejności:
\begin{enumerate}
  \item Uruchomić zewnętrznego i~wewnętrznego brokera wiadomości.
  \item Poczekać na prawidłowy start brokerów wiadomości.
  \item Uruchomić wymaganą liczbę nadzorców.
  \item Poczekać na prawidłowy start przynajmniej jednego z~nadzorców.
  \item Uruchomić serwer HTTP udostępniający panel administracyjny.
  \item Uruchomić wszystkie serwery wirtualizacji.
  \item Poczekać aż w~systemie znajdą się w~pełni uruchomione maszyny wszystkich zarejestrowanych typów.
\end{enumerate}

Po opisanych krokach system jest gotowy do użytkowania przez użytkowników.
Ważne jest, aby przed uruchomieniem jakiegokolwiek nadzorcy brokerzy wiadomości byli już prawidłowo zainicjalizowani i~gotowi na przyjęcie komunikacji.
W~przeciwnym wypadku aplikacja nadzorcy zakończy się z~błędem.
Serwer wirtualizacji także zakończy się z~błędem, jeżeli zabraknie brokerów wiadomości lub nadzorców.

\subsection{Parametry uruchomienia kontenera panelu administracyjnego}
Kontener zawierający panel administracyjny udostępnia aplikację poprzez serwer HTTP nginx (\url{https://www.nginx.com/}).
Do komunikacji wystawia on porty 80 (HTTP) oraz 443 (HTTPS), które należy przekierować na odpowiednie porty uruchamiającego systemu.

\subsubsection{Adres dostępu do nadzorców}
Aplikacja panelu administracyjnego oczekuje ustalonego adresu dostępu do jakiegokolwiek nadzorcy.
Przy uruchomieniu kontenera należy ustawić zmienną środowiskową \texttt{API\_URL} zgodnie z~opisem w~sekcji \ref{system_startup.admin_panel_conf}.

\subsubsection{Certyfikat SSL}
W~celu uruchomienia panelu w~trybie HTTPS należy przekazać certyfikat wraz ze zmodyfikowaną konfiguracją serwera nginx.
\begin{verbatim}
	-v {PATH_TO_CERT}:{PATH_TO_CERT_IN_CONTAINER}
	-v {PATH_TO_KEY}:{PATH_TO_KEY_IN_CONTAINER}
	-v {PATH_TO_CONF}:/etc/nginx/conf.d/default.conf
\end{verbatim}
gdzie \texttt{PATH\_TO\_CERT}, \texttt{PATH\_TO\_KEY} i~\texttt{PATH\_TO\_CONF} są ścieżkami bezwzględnymi na systemie uruchamiającym, a~\texttt{PATH\_TO\_CERT\_IN\_CONTAINER} oraz \texttt{PATH\_TO\_KEY\_IN\_CONTAINER} są ścieżkami bezwzględnymi zgodnymi ze ścieżkami zdefiniowanymi w~konfiguracji.

\subsection{Parametry uruchomienia kontenera nadzorcy}
Aplikacja nadzorcy udostępnia API poprzez serwer HTTP Kestrel.
Do komunikacji wystawia porty 5000 (HTTP) oraz 5001 (HTTPS), które można zmienić w~dostarczonej konfiguracji.
Należy przekierować je na odpowiednie porty uruchamiającego systemu.

\subsubsection{Plik konfiguracyjny}
Aplikacja nadzorcy poszukuje plików konfiguracyjnych w~lokalizacji \url{/overseer/config/}.
Należy przekazać folder zawierający konfiguracje z~systemu uruchamiającego (\texttt{PATH\_TO\_CONFIGS}) do wnętrza kontenera pod dokładnie tę lokalizację pod postacią woluminu.
\begin{verbatim}
	-v {PATH_TO_CONFIGS}:/overseer/config
\end{verbatim}

\subsubsection{Certyfikat SSL}
W~celu uruchomienia nadzorcy w~trybie HTTPS należy przekazać certyfikat SSL w~formacie \texttt{.pfx} przy pomocy woluminów oraz plik konfiguracyjny zgodny z~opisem z~rozdziału \ref{system_startup.overseer_conf}.
Przy konfiguracji używającej protokołu HTTPS i~braku certyfikatów aplikacja zakończy się z~błędem zaraz po uruchomieniu.

\begin{verbatim}
	-v {PATH_TO_CERT}:{PATH_TO_CERT_IN_CONTAINER}
\end{verbatim}
gdzie \texttt{PATH\_TO\_CERT} jest ścieżką bezwzględną do certyfikatu na systemie uruchamiającym oraz \texttt{PATH\_TO\_CERT\_IN\_CONTAINER} jest ścieżką bezwzględną wewnątrz kontenera zgodną z~konfiguracją.

\subsection{Parametry uruchomienia kontenera serwera wirtualizacji}
Serwer wirtualizacji zarządza maszynami wirtualnymi na systemie uruchamiającym.
Wymaga nietypowo podłączonych woluminów aby prawidłowo funkcjonować.

\subsubsection{Zasoby libvirta}
W~celu komunikacji z~usługą libvirta działającą na systemie uruchamiającym musimy przekazać gniazdo sieciowe do wnętrza kontenera.
Musi ono udawać, że jest usługą libvirta uruchomioną we wnętrzu kontenera.
Dodatkowo nowo tworzone maszyny powinny zapisywać się pomiędzy uruchomieniami kontenera.
W~tym celu trzeba przekazać także cały folder z~danymi libvirta.
Taki efekt można uzyskać przy pomocy woluminów przekazując
\begin{verbatim}
	-v /var/run/libvirt/libvirt-sock:/var/run/libvirt/libvirt-sock
	-v /var/lib/libvirt/:/var/lib/libvirt/
\end{verbatim}

\subsubsection{Zasoby vagranta}
Przy starcie kontenera zasoby vagranta są puste.
Oznacza to, że przy każdym pierwszym uruchomieniu maszyny wirtualnej będzie ona musiała być pobrana i~rozpakowana do zasobów libvirta.
Zabiera to czas i~przestrzeń dyskową.
Aby zapewnić trwałość Vagrant Boxów pomiędzy uruchomieniami, oraz zarządzanie nimi z~poziomu systemu uruchamiającego, należy przekazać do kontenera główny folder vagranta.
Folder domowy użytkownika wykonawczego powinien być wykorzystany do tego celu.

\begin{verbatim}
	-v ${HOME}/.vagrant.d/boxes/:/root/.vagrant.d/boxes/
\end{verbatim}
gdzie \texttt{HOME} to ścieżka do folderu domowego użytkownika wykonawczego.

\subsection{Uruchomienie maszyny wirtualnej w~trybie UEFI}

W~przypadku używania Vagrant Boxa skonfigurowanego do pracy z~UEFI trzeba przekazać libvirtowi odpowiedni obraz oprogramowania sprzętowego (najczęściej OVMF \parencite{ovmf}) oraz zmiennych NVRAM odpowiedzialnych za prawidłowe uruchomienie maszyny.
Zmienne NVRAM są zależne od obrazu maszyny wirtualnej.
Obraz zmiennych, odpowiadających Vagrant Boxowi podanemu w~pliku konfiguracyjnym, należy przekazać do wnętrza kontenera.
Ścieżkę do mapowanego obrazu można podać w~konfiguracji.

Dla obrazu \texttt{smogork/archlinux-rdp}, gdzie NVRAM dostarczony jest wewnątrz Vagrant Boxa, przekazanie parametru wygląda następująco:
\begin{verbatim}
	-v ${HOME}/.vagrant.d/boxes/smogork-VAGRANTSLASH-archlinux-rdp/0.2.1/libvirt/\
	nvram.fd:/app/nvram.fd
\end{verbatim}

\subsubsection{Plik konfiguracyjny}
Aplikacja serwera wirtualizacji poszukuje plików konfiguracyjnych w~lokalizacji \texttt{/app/config/docker-test/}.
Należy przekazać folder zawierający konfiguracje z~systemu uruchamiającego (\texttt{PATH\_TO\_CONFIGS}) do wnętrza kontenera pod dokładnie tą lokalizację jako wolumin.
\begin{verbatim}
	-v {PATH_TO_CONFIGS_DIR}:/app/config/
\end{verbatim}

Można także zmienić lokalizacje folderu z~konfiguracją ustawiając zmienną środowiskową \texttt{CONFIG}.
Może być ona względna do ścieżki \texttt{/app}.

Nie można zapomnieć o~wyjątku konfiguracji NLoga, którą trzeba przekazać do folderu \texttt{/app}.

\subsection{Przykład minimalnego systemu}
Uruchamiając kontenery z~modułami należy podać odpowiednie parametry, aby aplikacje wewnątrz pracowały prawidłowo.
Przykładowy minimalny system wraz z~parametrami można zobaczyć w~module \texttt{demonstration} \parencite{ocd-demo}.
Znajduje się tam system składający się z~jednego brokera wiadomości (zewnętrzna i~wewnętrzna komunikacja w~jednym), panelu administracyjnego, jednego nadzorcy i~jednego serwera wirtualizacji. Konta użytkowników zdefiniowane są w~systemie katalogowym, a~ich katalogi domowe udostępniane poprzez serwer NFS. W~celu umożliwienia uruchomienia większej ilości nadzorców stworzony został prosty serwer dostępowy.
System zawarty w~tym repozytorium udostępnia pełną funkcjonalność.
Można znaleźć tam przykłady zarówno konfiguracji modułów jak i~parametrów startowych dla kontenerów.

\end{document}