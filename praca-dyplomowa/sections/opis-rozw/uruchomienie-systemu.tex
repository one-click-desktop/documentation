\documentclass[../opis-rozwiazania.tex]{subfiles}

\begin{document}
\label{system_startup}

W tym podrozdziale zakładamy, że każdy system operacyjny został skonfigurowany poprawnie zgodnie z opisem w rozdziale \ref{system_requirements}.

\subsection{Pozyskanie aplikacji klienckiej}
\label{system_startup.client_obtaining}
Zalecanym sposobem pozyskania aplikacji klienckiej jest udanie się do sekcji z oficjalnymi wydaniami \parencite{ocd-client-releases} modułu aplikacji klienckiej oraz pobranie najnowszej wersji dla wybranego systemu operacyjnego.
Przy pierwszym uruchomieniu trzeba pobrać archiwum plików zawierające aplikację oraz plik konfiguracyjny.
Bez pliku konfiguracyjnego aplikacja nie uruchomi się.

\subsection{Zbudowanie kontenerów}
\label{system_startup.containers}

Moduły: panelu administracyjnego, nadzorcy i serwera wirtualizacji można uruchomić pod postacią kontenera dockerowego.
Jest to zalecany sposób uruchamiania tych trzech modułów.

\subsubsection{Ujednolicony sposób budowania kontenerów}
Każdy z tych 3 modułów ma przygotowany skrypt \texttt{build.sh}, który powinien prawidłowo zbudować kontener.
Skrypt ten wykona polecenie do budowania i oznaczy kontener odpowiednią nazwą.
Należy wykonać go zawsze z poziomu głównego folderu repozytorium modułu.
Każdy kontener przy starcie wykona skrypt \texttt{assets/entry\_point.sh}.

\subsubsection{Budowanie kontenera serwera wirtualizacji}
Kontener serwera wirtualizacji wymaga specjalnie przygotowanego wcześniej kontenera zawierającego wszystkie potrzebne zależności do uruchomienia aplikacji.
Aby go zbudować należy skorzystać z pliku \texttt{\seqsplit{runtime\_container/Dockerfile}} i oznaczyć go nazwą \texttt{\seqsplit{click-desktop/virtualization-server-runtime}}.
Kontener zawierający aplikację w czasie budowania będzie oczekiwał, ze taki obraz istnieje.

Po zbudowaniu kontenera z zależnościami można przystąpić do budowania kontenera głównego.
Należy do tego celu skorzystać z dostarczonego pliku \texttt{Dockerfile} w głównym folderze repozytorium.

Przy uruchomieniu skryptu \texttt{build.sh} kontener jest budowany z nazwą \texttt{one-click-desktop/virtualization-server}.

\subsubsection{Budowanie kontenera nadzorcy i panelu administratora}
W przypadku tych dwóch modułów nie trzeba wykonać żądnych dodatkowych przygotowań.
Wystarczy skorzystać z dostarczonego pliku \texttt{Dockerfile} w głównym folderze repozytorium.

Przy uruchomieniu skryptu \texttt{build.sh} kontenery są budowane z nazwami odpowiednio \texttt{one-click-desktop/overseer} oraz \texttt{\seqsplit{one-click-desktop/admin-panel}}.

\subsection{Budowanie modułów}
Każdy z modułów posiada dokładną instrukcję budowania w pliku \texttt{README.md}.
Uruchamianie zbudowanych modułów poleca się tylko w przypadku rozwoju aplikacji.
Przy wdrażaniu systemu najszybciej i najbezpieczniej jest skorzystać z metod opisanych w sekcjach \ref{system_startup.client_obtaining} i \ref{system_startup.containers}.

\subsubsection{Moduły napisane w technologii .NET}
W przypadku budowania modułu nadzorcy i serwera wirtualizacji potrzebne są narzędzia deweloperskie .NET 5.0.
W trakcie budowania aplikacji wszystkie użyte biblioteki zostaną pobrane przy użyciu menadżera pakietów Nuget.
Przy uruchomieniu serwera wirtualizacji potrzebny jest, poza wymaganiami zdefiniowanymi w rozdziale \ref{system_requirements.virtsrv_rquirements}, zainstalowany \texttt{Vagrant} wraz z wtyczka \texttt{vagrant-libvirt} w wersji przynajmniej 0.7.0.

\subsubsection{Moduły napisane w technologii Node}
W przypadku budowania modułów aplikacji klienckiej oraz panelu administracyjnego potrzebny będzie pakiet \texttt{Node}. Przed zbudowaniem aplikacji należy pobrać wszystkie użyte biblioteki przy użyciu menadżera pakietów \texttt{npm}.
Są one zdefiniowane w projekcie aplikacji i zostaną pobrane automatycznie.

\subsection{Konfiguracja aplikacji klienckiej}
Aplikacja kliencka wymaga pliku konfiguracyjnego o nazwie \texttt{config.json} w tym samym folderze co plik wykonywalny.
W pliku konfiguracyjnym użytkownik musi podać:
\begin{itemize}
  \item Adres jednego z nadzorców lub serwera dostępowego (\texttt{basePath}) - adres musi być w formacie URI.
  \item Adres zewnętrznego brokera wiadomości (\texttt{rabbitPath}) - adres musi być w formacie URI.
  \item Czy aplikacja powinna podczas połączenia RDP używać danych dostępowych takich samych jak przy logowaniu do systemu (\texttt{useRdpCredentials}) - \texttt{true} albo \texttt{false}.
  \item Czy aplikacja powinna uruchamiać zintegrowanego klienta RDP (\texttt{startRdp}) - \texttt{true} albo \texttt{false}. Przydaje się to przy wykorzystaniu innego niż zalecany klient RDP. Aplikacja po uzyskaniu sesji wyświetli dane dostępowe do przypisanej maszyny wirtualnej.
\end{itemize}

\subsection{Konfiguracja panelu administracyjnego}
\label{system_startup.admin_panel_conf}
Przy uruchomieniu panelu administracyjnego trzeba przekazać adres jednego z nadzorców.
Aby tego dokonać należy ustawić zmienną środowiskową \texttt{API\_URL} na adres dostępowy jednego z nadzorców lub serwera dostępowego.

\subsection{Konfiguracja nadzorcy}
\label{system_startup.overseer_conf}
Nadzorca posiada wiele parametrów związanych z działaniem całego systemu.
Kontroluje on między innymi czas oczekiwania na powrót użytkownika do porzuconej sesji.
Wszystkie parametry należy przekazać mu poprzez plik konfiguracyjny.

Nazwa pliku konfiguracyjnego musi spełniać wzorzec \texttt{\seqsplit{appsettings.\$\{Nazwa\_srodowiska\}.ini}}.
W przypadku uruchamiania aplikacji wewnątrz kontenera ustawione jest środowisko o nazwie \texttt{Production}.
Aplikacja nadzorcy domyślnie przeszukuje folder \texttt{./config} w poszukiwaniu plików konfiguracyjnych.
Można zmienić ścieżkę do folderu konfiguracyjnego poprzez parametr uruchomienia \texttt{-c path/to/other/config/folder/}.

W pliku konfiguracyjnym nadzorca poszukuje 2 specjalnych sekcji:
\begin{itemize}
  \item \texttt{JwtSettings} - przechowuje parametr wykorzystywany do generowania unikatowych tokenów autoryzacyjnych
  \item \texttt{OneClickDesktop} - przechowuje parametry związane z działaniem systemu OneClickDesktop.
\end{itemize}
Poza tymi sekcjami można tam umieścić dowolne sekcje, które będzie przetwarzać ASP.NET \parencite{asp-conf}.
Jednak warte uwagi są następujące sekcje:
\begin{itemize}
  \item \texttt{Logging} - parametry loggera dostarczonego przez Microsoft (\url{https://www.nuget.org/packages/microsoft.extensions.logging}).
  \item \texttt{Kestrel} - parametry serwera HTTP udostępniającego aplikację. W wypadku aplikacji nadzorcy bez ustawienia certyfikatu nie zadziała ona w trybie HTTPS.
\end{itemize}
\noindent
Pozostają jeszcze 2 ważne parametry bez sekcji zaimplementowane przez ASP.NET:
\begin{itemize}
  \item \texttt{AllowedHosts} - lista adresów z których zapytania będą obsługiwane.
  \item \texttt{urls} - lista adresów, na których aplikacja będzie nasłuchiwać zapytań.
\end{itemize}

\subsubsection{Parametry sekcji \texttt{JwtSettings}}
Sekcja ta zawiera jedynie parametr \texttt{Secret}. Należy go ustawić na dowolny napis.

\subsubsection{Parametry sekcji \texttt{OneClickDesktop}}
Sekcja zawiera parametry związane z komunikacją pomiędzy jednostkami systemu oraz kilka parametrów określających interwały czasowe zdarzeń.
W dostarczonej przykładowej konfiguracji wykomentowane wartości oznaczają wartości domyślne.
\begin{itemize}
  \item \texttt{OverseerId} - identyfikator nadzorcy używany w komunikacji poprzez wewnętrznego brokera wiadomości. Powinna być unikatowa w skali jednej instancji systemu OneClickDesktop. Domyślnie przyjmuje wartość \texttt{overseer-test}.
  \item \texttt{RabbitMQHostname} - adres dostępowy wewnętrznego brokera wiadomości. Domyślnie przyjmuje wartość \texttt{localhost}. Bez działającego procesu brokera pod tym adresem aplikacja wyłączy się z błędem zaraz po uruchomieniu.
  \item \texttt{RabbitMQPort} - port dostępowy wewnętrznego brokera wiadomości. Domyślnie przyjmuje wartość \texttt{5672}. Bez działającego procesu brokera pod tym portem aplikacja wyłączy się z błędem zaraz po uruchomieniu.
  \item \texttt{ModelUpdateInterval} - ilość sekund co ile nadzorca prosi serwery wirtualizacji o aktualizacje modelu. Domyślnie przyjmuje wartośc \texttt{60}. Zalecane jest aby wartość tego parametru opisywała czas od 30 do 60 sekund.																												%wtf? ą nie działa we wnętrzu {}
  \item \texttt{DomainShutdownTimeout} - ilość minut jaka musi upłynąć od oznaczenia maszyny wirtualnej stanem \textit{Oczekiwanie na wyłączenie maszyny} do jej wyłączenia. Domyślnie przyjmuje wartość \texttt{15}.
  \item \texttt{DomainShutdownCounterInterval} - ilość sekund co ile nadzorca sprawdza czy maszyna wirtualna nadal oczekuje na zamknięcie. Zaleca się, aby ten czas dzielił czas z parametru \texttt{DomainShutdownTimeout} na równe części. Takich części nie powinno być mniej niż 10, ale także nie więcej niż 100. Każde takie sprawdzenie zużywa czas procesora.
\end{itemize}

\subsubsection{Przykładowa konfiguracja}

Konfiguracja ta umożliwia użycie protokołu HTTPS dzięki przekazaniu zabezpieczonego hasłem certyfikatu oraz nasłuchiwaniu na adresie obsługującym ten protokół.

\begin{verbatim}
	AllowedHosts=*
	urls=http://*:5000;https://*:5001

	[Kestrel:Certificates:Default]
	Password=password
	Path=/overseer/overseer.pfx

	[JwtSettings]
	Secret=MySecretForJWTTokensPleaseChangeMe

	[Logging.LogLevel]
	Default=Information
	Microsoft=Warning
	Microsoft.Hosting.Lifetime=Information

	[OneClickDesktop]
	OverseerId=overseer-instance-123
	RabbitMQHostname=1.2.3.4
	RabbitMQPort=5678
	ModelUpdateInterval=60
	DomainShutdownTimeout=15
	DomainShutdownCounterInterval=30
\end{verbatim}

\subsection{Konfiguracja serwera wirtualizacji}
\label{system_startup.virtsrv_conf}
Serwer wirtualizacji posiada wiele parametrów związanych z zasobami przekazanymi do dyspozycji systemu.
Wszystkie parametry należy przekazać mu poprzez pliki konfiguracyjne.

Pliki te muszą znaleźć się w jednym folderze.
Głównym plikiem konfiguracyjnym jest \texttt{virtsrv.ini}.
Zawiera on parametry związane z komunikacją wewnątrz i na zewnątrz systemu.
Dodatkowo znajdują się tam informacje o udostępnionych zasobach dla serwera wirtualizacji.
Dodatkowe pliki konfiguracyjne reprezentują typy maszyn wirtualnych uruchamianych na tym serwerze wirtualizacji.
Jedynym odstępstwem od reguły jest plik konfiguracyjny dla pakietu NLog.
Musi on się znajdować w tym samym folderze co aplikacja oraz nazywać się \texttt{NLog.config}.

Aplikacja serwera wirtualizacji domyślnie przeszukuje folder \texttt{./config} w poszukiwaniu plików konfiguracyjnych.
Można zmienić ścieżkę do folderu konfiguracyjnego poprzez parametr uruchomienia \texttt{-c\ path/to/other/config/folder/}.

\subsubsection{Główny plik konfiguracyjny}
W głównym pliku konfiguracyjnym możemy wyróżnić 2 sekcje:
\begin{itemize}
  \item \texttt{OneClickDesktop} - przechowuje parametry związane z działaniem systemu OneClickDesktop.
  \item \texttt{ServerResources} - przechowuje parametry określające wszystkie udostępnione zasoby.
\end{itemize}

W sekcji \texttt{OneClickDesktop} występują parametry:
\begin{itemize}
  \item \texttt{VirtualizationServerId} - identyfikator serwera wirtualizacji używany w komunikacji poprzez wewnętrznego brokera wiadomości. Powinna być unikatowa w skali jednej instancji systemu OneClickDesktop. Domyślnie przyjmuje wartość \texttt{virtsrv-test}.
  \item \texttt{OversserCommunicationShutdownTimeout} - po upływie tylu sekund bez komunikacji od jakiegokolwiek nadzorcy serwer wirtualizacji uznaje, że zabrakło nadzorców w systemie. Nastąpi wtedy wyłączenie aplikacji serwera wirtualizacji. Domyślnie parametr przyjmuje wartość \texttt{120}. Zaleca się aby wartość parametru była większa niż dwukrotność parametru \texttt{ModelUpdateInterval} z konfiguracji nadzorcy opisanej w sekcji \ref{system_startup.overseer_conf}.
  \item \texttt{LibvirtUri} - adres do połaczenia z usługą libvirta. Vagrant jak i biblioteka libvirta skorzysta z tego adresu do połaczenia. Specyfikacja libvirta dokładnie opisuje format tego adresu \parencite{libvirt-uri}.
  \item \texttt{VagrantFilePath} - ścieżka do specjalnie przygotowanego pliku wsadowego dla Vagranta. Wykorzystywany jest przez system do uruchamiania i wyłączania maszyn wirtualnych. Aplikacja nie zadziała prawidłowo bez ustawionej wartości parametru.
  \item \texttt{PostStartupPlaybook} - skrypt wykonywany przy pomocy Ansible'a zaraz po uruchomieniu maszyny wirtualnej. Koniecznie należy przetestować czy konfiguracja wykonuje się prawidłowo. Przy każdym błędzie wykonania skryptu maszyna wirtualna zostanie usunięta. Domyślnie przyjmuje wartość \texttt{res/poststartup\_playbook.yml}.
  \item \texttt{VagrantboxUri} - identyfikator szablonowej maszyny wirtualnej. Musi ona spełniać szereg wymogów opisanych w sekcji \ref{system_requirements.vagrant_box}. W przeciwnym wypadku system nie będzie w stanie uruchomić takiego szablonu. Aplikacja nie zadziała prawidłowo bez ustawionej wartości parametru.
  \item \texttt{BridgeInterfaceName} - nazwa interfejsu sieciowego w skonfigurowanego w trybie bridge, do którego zostanie podłączona każda uruchomiona maszyna wirtualna. Domyślnie przyjmuje wartość \texttt{br0}.
  \item \texttt{BridgedNetwork} - adres sieci, do której podłączona zostanie maszyna wirtualna poprzez \texttt{BridgeInterfaceName}, podany w formacie CIDR. Jeżeli maszyna nie będzie posiadać adresu z podanej sieci po uruchomieniu to zostanie wyłączona. Aplikacja nie zadziała prawidłowo bez ustawionej wartości parametru.
  \item \texttt{InternalRabbitMQHostname} - adres dostępowy wewnętrznego brokera wiadomości. Domyślnie przyjmuje wartość \texttt{localhost}. Bez działającego procesu brokera pod tym adresem aplikacja wyłączy się z błędem zaraz po uruchomieniu.
  \item \texttt{InternalRabbitMQPort} - port dostępowy wewnętrznego brokera wiadomości. Domyślnie przyjmuje wartość \texttt{5672}. Bez działającego procesu brokera pod tym portem aplikacja wyłączy się z błędem zaraz po uruchomieniu.
  \item \texttt{ExternalRabbitMQHostname} - adres dostępowy zewnętrznego brokera wiadomości. Domyślnie przyjmuje wartość \texttt{localhost}. Bez działającego procesu brokera pod tym adresem aplikacja wyłączy się z błędem zaraz po uruchomieniu.
  \item \texttt{ExternalRabbitMQPort} -  port dostępowy zewnętrznego brokera wiadomości. Domyślnie przyjmuje wartość \texttt{5673}. Bez działającego procesu brokera pod tym portem aplikacja wyłączy się z błędem zaraz po uruchomieniu.
  \item \texttt{ClientHeartbeatChecksForMissing} - liczba nieudanych sprawdzeń czy aplikacja kliencka nadal jest połączona do maszyny wirtualnej. Po tej liczbie prób maszyna wirtualna oznaczana jest jako oczekująca na zamknięcie. Domyslnie przyjmuje wartośc \texttt{2}.
  \item \texttt{ClientHeartbeatChecksDelay} - czas w milisekundach pomiędzy sprawdzeniami, czy aplikacja kliencka nadal jest połączona do maszyny wirtualnej. Domyślnie przyjmuje wartość \texttt{10000}
\end{itemize}

W sekcji \texttt{ServerResources} występują parametry:
\begin{itemize}
  \item \texttt{Cpus} - liczba wątków, które może wykorzystać system. Domyślnie przyjmuje wartość \texttt{2}.
  \item \texttt{Memory} - ilość pamięci operacyjnej w MiB jaką może wykorzystać system. Domyślnie przyjmuje wartość \texttt{2048}.
  \item \texttt{Storage} - ilość przestrzeni dyskowej w GiB jaką może wykorzystać system. Domyślnie przyjmuje wartość \texttt{100}.
  \item \texttt{GPUsCount} - ilość kart graficznych przekazanych do systemu. Domyslnie przyjmuje wartośc \texttt{0}.
  \item \texttt{MachineTypes} - lista typów maszyn wirtualnych. Każda nazwa typu jest oddzielona od sąsiednich przecinkiem. Nazwa typu musi składać się ze znaków opisanych następującym wyrażeniem regularnym \texttt{[a-zA-Z0-9\textbackslash-]}. Dla każdej z nazw wyszukiwany jest plik konfiguracyjny o o nazwie zaczynającej się nazwą typu i kończącą się \texttt{\_template.ini}.
\end{itemize}

Gdy $n = \texttt{GPUsCount}$ jest większy od 0, wtedy w pliku muszą znaleźć się sekcje od \texttt{ServerGPU.1} do \texttt{ServerGPU.$n$} włącznie.
W przeciwnym wypadku serwer wirtualizacji zakończy się z błędem zaraz po uruchomieniu.

Każda z tych sekcji zawiera parametr \texttt{AddressCount}, który określa jak duża jest grupa IOMMU w której znajduje się przekazywane urządzenie PCI.
W przypadku $m = \texttt{AddressCount}$ w tej sekcji muszą znaleźć się parametry od \texttt{Address\_1} do \texttt{Address\_$m$} włącznie.
W przeciwnym wypadku serwer wirtualizacji zakończy się z błędem zaraz po uruchomieniu.
Każdy z tych parametrów opisuje adres pojedynczego urządzenia na magistrali PCI.
Adres na magistrali PCI musi zostać opisany w formacie uzyskanym z polecenia \texttt{lspci} - \texttt{\$\{domain:4\}:\$\{bus:2\}:\$\{slot:2\}.\$\{function:1\}}.

Przykładowa zawartość głównego pliku konfiguracyjnego:
\begin{verbatim}
	[OneClickDesktop]
	VirtualizationServerId=virtsrv-instance-123
	OversserCommunicationShutdownTimeout=120
	LibvirtUri=qemu:///system
	VagrantFilePath=res/Vagrantfile
	PostStartupPlaybook=res/poststartup_playbook.yml
	VagrantboxUri=smogork/archlinux-rdp

	BridgeInterfaceName=br0
	BridgedNetwork=1.2.3.0/24

	InternalRabbitMQHostname=1.2.3.4
	InternalRabbitMQPort=5678

	ExternalRabbitMQHostname=1.2.3.4
	ExternalRabbitMQPort=8765

	ClientHeartbeatChecksForMissing=2
	ClientHeartbeatChecksDelay=10000

	[ServerResources]
	Cpus=6
	Memory=4096
	Storage=200
	GPUsCount=1
	MachineTypes=cpu,cpu-memory

	[ServerGPU.1]
	AddressCount=2
	Address_1=0000:03:00.0
	Address_2=0000:03:00.1
\end{verbatim}

\subsubsection{Pliki konfiguracyjne opisujące typy maszyn wirtualnych}
W pliku opisującym typy zawarte jest ile potrzeba zasobów aby uruchomić maszynę tego typu.

Każdy typ ma przypisaną nazwę. Oznaczmy ją jako \texttt{\$\{template\_name\}}.
Nazwa typu powinna być unikatowa na skale systemu.
Jeżeli chcemy udostępnić ten sam typ na wielu serwerach wirtualizacji, należy nadać mu takie same zasoby.
System może nieprawidłowo zarządzać zasobami, gdy uzyska 2 wzorce zasobów tego samego typu maszyny.
Plik opisujący typ musi mieć nazwę \texttt{\$\{template\_name\}\_template.ini}.
Plik konfiguracyjny dla wybranego typu musi znajdować się w folderze z pozostałymi plikami konfiguracyjnymi.
Jeżeli dla jakiejkolwiek nazwy typu aplikacja nie znajdzie pliku konfiguracyjnego to zakończy się z błędem zaraz po uruchomieniu.

W znalezionym pliku musi być sekcja o nazwie \texttt{\$\{template\_name\}\_template}.
Zwiera ona parametry:
\begin{itemize}
  \item \texttt{HumanReadableName} - nazwa reprezentowana w aplikacji klienckiej dla użytkownika. Domyślnie przyjmuje wartość \texttt{\$\{template\_name\}}.
  \item \texttt{Cpus} - liczba wątków potrzebna do uruchomienia maszyny tego typu. Domyślnie parametr przyjmuje wartość \texttt{2}.
  \item \texttt{Memory} - ilość pamięci operacyjnej w MiB potrzebnej do uruchomienia maszyny tego typu. Domyślnie parametr przyjmuje wartość \texttt{512}.
  \item \texttt{Storage} - ilość przestrzeni dyskowej w GiB potrzebnej do uruchomienia maszyny tego typu. Domyślnie parametr przyjmuje wartość \texttt{20}.
  \item \texttt{AttachGpu} - informacja czy maszyna tego typu przy uruchomieniu powinna mieć przekazaną kartę graficzną. Domyślnie parametr przyjmuje wartość \texttt{false}.
\end{itemize}

Przykładowa zawartość pliku konfiguracyjnego typu maszyny:

\begin{verbatim}
	[gpu_template]
	HumanReadableName=Efficiency machine with GPU attached
	Cpus=1
	Memory=2048
	Storage=55

	AttachGpu = true
\end{verbatim}

\subsection{Procedura uruchomienia}
\label{system_startup.procedure}
Prawidłowy start systemu powinien zachować następująca kolejność:
\begin{enumerate}
  \item Uruchomić zewnętrznego i wewnętrznego brokera wiadomości.
  \item Poczekać na prawidłowy start brokerów wiadomości.
  \item Uruchomić wymaganą liczbę nadzorców.
  \item Poczekać na prawidłowy start przynajmniej jednego z nadzorców.
  \item Uruchomić serwer HTTP udostępniający panel administracyjny.
  \item Uruchomić wszystkie serwery wirtualizacji.
  \item Poczekać aż w systemie znajdą się w pełni uruchomione maszyny wszystkich zarejestrowanych typów.
\end{enumerate}

Po opisanych krokach system jest gotowy do użytkowania przez użytkowników.
Ważne jest, aby przed uruchomieniem jakiegokolwiek nadzorcy brokerzy wiadomości byli już prawidłowo zainicjalizowani i gotowi na przyjęcie komunikacji.
W przeciwnym wypadku aplikacja nadzorcy zakończy się z błędem.
Serwer wirtualizacji także zakończy się z błędem, jeżeli zabraknie brokerów wiadomości.
Dodatkowo, jeżeli nie będzie żadnego nadzorcy nasłuchującego poprzez wewnętrznego brokera wiadomości, serwer wirtualizacji zgłosi błąd i zakończy pracę.

\subsection{Parametry uruchomienia kontenera panelu administracyjnego}
Kontener zawierający panel administratory udostępnia aplikacje poprzez serwer HTTP \texttt{nginx}.
Do komunikacji wystawia on porty 80 (HTTP) oraz 443 (HTTPS).
Należy przekierować je na odpowiednie porty uruchamiającego systemu.

\subsubsection{Adres dostępu do nadzorców}
Aplikacja panelu administracyjnego oczekuje ustalonego adresu dostępu do jakiegokolwiek nadzorcy.
Przy uruchomieniu kontenera należy ustawić zmienną środowiskową \texttt{API\_URL} zgodnie z opisem w sekcji \ref{system_startup.admin_panel_conf}.

\subsubsection{Certyfikat SSL}
W celu uruchomienia panelu w trybie HTTPS należy przekazać zmienioną konfigurację serwera nginx oraz parę klucza z certyfikatem pod postacią woluminu.
\begin{verbatim}
	-v {PATH_TO_CERT}:{PATH_TO_CERT_IN_CONTAINER}
	-v {PATH_TO_KEY}:{PATH_TO_KEY_IN_CONTAINER}
	-v {PATH_TO_CONF}:/etc/nginx/conf.d/default.conf
\end{verbatim}
gdzie \texttt{PATH\_TO\_CERT}, \texttt{PATH\_TO\_KEY} i \texttt{PATH\_TO\_CONF} są ścieżkami bezwzględnymi na systemie uruchamiającym, a \texttt{PATH\_TO\_CERT\_IN\_CONTAINER} oraz \texttt{PATH\_TO\_KEY\_IN\_CONTAINER} są ścieżkami bezwzględnymi w kontenerze podanymi w przyłączonej konfiguracji.

\subsection{Parametry uruchomienia kontenera nadzorcy}
Aplikacja nadzorcy udostępnia API poprzez serwer HTTP Kestrel.
Do komunikacji wystawia porty 5000 (HTTP) oraz 5001 (HTTPS), które można zmienić w dostarczonej konfiguracji.
Należy przekierować je na odpowiednie porty uruchamiającego systemu.

\subsubsection{Plik konfiguracyjny}
Aplikacja nadzorcy poszukuje plików konfiguracyjnych w lokalizacji \url{/overseer/config/}.
Należy przekazać folder zawierający konfiguracje z systemu uruchamiającego (\texttt{PATH\_TO\_CONFIGS}) do wnętrza kontenera pod dokładnie tę lokalizację pod postacią woluminu.
\begin{verbatim}
	-v {PATH_TO_CONFIGS}:/overseer/config
\end{verbatim}

\subsubsection{Certyfikat SSL}
W celu uruchomienia nadzorcy w trybie HTTPS należy przekazać certyfikat SSL w formacie \texttt{.pfx} przy pomocy woluminów oraz odpowiednio przygotowany plik konfiguracyjny tak jak opisano w rozdziale \ref{system_startup.overseer_conf}.
Gdy zabraknie certyfikatów, a konfiguracja pokazuje, że ma się rozpocząć nasłuchiwanie w trybie HTTPS, nadzorca zakończy się z błędem zaraz po uruchomieniu.

\begin{verbatim}
	-v {PATH_TO_CERT}:{PATH_TO_CERT_IN_CONTAINER}
\end{verbatim}
gdzie \texttt{PATH\_TO\_CERT} jest ścieżką bezwzględną do certyfikatu na systemie uruchamiającym oraz \texttt{PATH\_TO\_CERT\_IN\_CONTAINER} jest ścieżką bezwzględną wewnątrz kontenera podaną w konfiguracji.

\subsection{Parametry uruchomienia kontenera serwera wirtualizacji}
Serwer wirtualizacji zarządza maszynami wirtualnymi na systemie uruchamiającym.
Wymaga dość nietypowo podłączonych woluminów aby prawidłowo funkcjonować.

\subsubsection{Zasoby libvirta}
Aby komunikować się z usługa libvirta uruchomioną na systemie uruchamiającym potrzebujemy gniazda sieciowego przekazanego do wnętrza kontenera.
Musi ono udawać, że jest uruchomioną usługą libvirta we wnętrzu kontenera.
Dodatkowo nowo tworzone maszyny powinny zapisywać się pomiędzy uruchomieniami kontenera.
W tym celu trzeba przekazać także cały folder z danymi libvirta.
Taki efekt można uzyskać przy pomocy woluminów przekazując
\begin{verbatim}
	-v /var/run/libvirt/libvirt-sock:/var/run/libvirt/libvirt-sock
	-v /var/lib/libvirt/:/var/lib/libvirt/
\end{verbatim}

\subsubsection{Zasoby vagranta}
Przy starcie kontenera zasoby vagranta są puste.
Oznacza to, że przy każdym pierwszym uruchomieniu maszyny wirtualnej będzie ona musiała być pobrana i rozpakowana do zasobów libvirta.
Zabiera to czas i przestrzeń dyskową.
Aby zapewnić trwałość Vagrant Boxów pomiędzy uruchomieniami, oraz zarządzanie nimi z poziomu systemu uruchamiającego, należy przekazać do kontenera główny folder vagranta.
Folder domowy użytkownika wykonawczego powinien być wykorzystany do tego celu.

\begin{verbatim}
	-v ${HOME}/.vagrant.d/boxes/:/root/.vagrant.d/boxes/
\end{verbatim}
gdzie \texttt{HOME} to ścieżka do folderu domowego użytkownika wykonawczego.

\subsubsection{Plik konfiguracyjny}
Aplikacja serwera wirtualizacji poszukuje plików konfiguracyjnych w lokalizacji \texttt{/app/config/docker-test/}.
Należy przekazać folder zawierający konfiguracje z systemu uruchamiającego (\texttt{PATH\_TO\_CONFIGS}) do wnętrza kontenera pod dokładnie tą lokalizację pod postacią woluminu.
\begin{verbatim}
	-v {PATH_TO_CONFIGS_DIR}:/app/config/
\end{verbatim}

Można także zmienić lokalizacje folderu z konfiguracją ustawiając zmienną środowiskową \texttt{CONFIG}.
Może być ona względna do ścieżki \texttt{/app}.

Nie można zapomnieć o wyjątku konfiguracji NLoga, którą trzeba przekazać do folderu \texttt{/app}.

\subsection{Przykład minimalnego systemu}
Uruchamiając kontenery z modułami należy podać odpowiednie parametry, aby aplikacje wewnątrz pracowały prawidłowo.
Przykładowy minimalny system wraz z parametrami można zobaczyć w module \texttt{demonstration} \parencite{ocd-demo}.
Znajduje się tam system składający się z jednego brokera wiadomości (zewnętrzna i wewnętrzna komunikacja w jednym), panelu administracyjnego, jednego nadzorcy i jednego serwera wirtualizacji.
Można znaleźć tam przykłady zarówno konfiguracji modułów jak i parametrów startowych dla kontenerów.

\end{document}