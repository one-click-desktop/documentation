\documentclass[../opis-rozwiazania.tex]{subfiles}

\begin{document}
\label{technologies}

\subsection{Technologie realizacji strony klienckiej}
%[LINK]Dodac strony do technologii
Realizacja panelu administracyjnego oraz aplikacji klienckiej, przez które użytkownicy i administratorzy korzystają z systemu, zostały wykonane na platformie programistycznej Angular.
Wykorzystaliśmy ją z powodu znajomości technologii, więc mogliśmy sprawnie zaimplementować aplikacje.
W każdym przypadku skorzystaliśmy z języka TypeScript, który dobrze integruje się z Angularem dzięki pośredniej kompilacji do JavaScriptu.

Aplikacja kliencka uruchamiana jest na systemie operacyjnym użytkownika poprzez technologie Electron, która umożliwia wyświetlania aplikacji Angulara w postaci okna.
W wielkim uproszczeniu Electron korzysta z silnika Chrome'a aby uruchomić aplikację jako stronę internetową i wyświetlić ją jak inne okna w systemie.
Ułatwiło nam to stworzenie aplikacji od razu na wiele systemów operacyjnych.
Dodatkowym plusem korzystania z Electrona jest możliwość odwołania do silnika NodeJS, aby zwiększyć możliwości dostępu do zasobów systemu operacyjnego.
Wykorzystaliśmy go do uruchamiania podprocesu klienta RDP żeby po uzyskaniu sesji od razu zestawić połączenie z wybrana maszyną wirtualną.

\subsection{Technologie realizacji strony serwerowej}
%[LINK] Dodac strony do technologii
W przypadku aplikacji nadzorcy oraz serwera wirtualizacji zdecydowaliśmy się na środowisko uruchomieniowe .NET 5.0 z językiem C\#.
Ekosystem .NET jest przez nas dobrze poznany i daje nam możliwości aby ułatwić oraz przyśpieszyć naszą pracę.
Głównym ułatwieniem była duża baza bibliotek zgromadzona w menadżerze pakietów Nuget, do którego dodaliśmy też własne pakiety, aby wydzielić wspólną część kodu aplikacji serwerowych.

Do zarządzania i monitorowania maszyn wirtualnych skorzystaliśmy z libvirta.
W serwerze wirtualizacji wykorzystaliśmy rozszerzoną przez nas bibliotekę odwołującą się do oryginalnych bibliotek libvirta.
Jest to technologie powszechnie znana i dobrze przez nas poznana. Był to najprostszy sposób do realizacji zarządcy maszyn wirtualnych.
Dodatkowo współpracuje ona także z technologia Vagrant, która umożliwia proste powielanie maszyn wirtualnych z wzorca dostarczonego przez użytkownika.
Ułatwi ona administratorom tworzenie obrazów uruchamianych systemów operacyjnych.
Do dodatkowych konfiguracji takich maszyn wirtualnych użyliśmy technologii Ansible.
Wybraliśmy ją, ponieważ jest znacznie mniej bezawaryjna niż konfigurowanie systemu przez skrypty powłoki.
Oprócz tego skorzystaliśmy z niej do konfiguracji komputerów przed uruchomieniem pisanego przez nas systemu.

\subsection{Technologie testowania}
%[LINK] Dodac strony do technologii
Do testów jednostkowych aplikacji od strony klienckiej skorzystaliśmy z platformy Jest.
Ułatwia ona pisanie testów przy niepełnym obrazie systemu. W przypadku rozbudowanej komunikacji naszego systemu było to bardzo przydatne.
W przypadku aplikacji serwerowych skorzystaliśmy z platformy NUnit.
Znaliśmy ją z wcześniejszych projektów, więc mogliśmy od razu przystąpić do pracy.
Do testów komunikacji poprzez brokera wiadomości oraz integracji z libvirtem także skorzystaliśmy z NUnita.

Do testów integracyjnych aplikacji Angularowych korzystaliśmy z platformy Cypress.
Zdecydowaliśmy się na nią, ponieważ wprowadzała nowa funkcje zrzutu zmiennych przy błędnym teście.
Wcześniej znane nam rozwiązania nie posiadały takich funkcji, przez co bardzo trudno było szukać problemów w testach.
Dodatkowe testy API wystawianego przez aplikacje nadzorcę wykonaliśmy Postmanem.

\subsection{Utrzymanie kodu}
%[LINK] Dodac strony do technologii
Do sprawnego zarządzania projektem wykorzystaliśmy strukturę wielu repozytoriów dostępnych na stronie github.
Wspólne części kodu wydzielaliśmy w postaci pakietów NPM oraz Nuget.
Aby zautomatyzować proces publikacji pakietów skorzystaliśmy z e środowiska ciągłej integracji Github Actions.
Umożliwiło to praktycznie bezproblemowa realizację komunikacji pomiędzy nadzorcą a aplikacją kliencką i panelem administratora.
Tworzyliśmy automatycznie generowane pakiety z częściową implementacja obiektów transportowych oraz struktury zapytań.

\end{document}