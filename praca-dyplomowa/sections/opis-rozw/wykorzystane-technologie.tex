\documentclass[../opis-rozwiazania.tex]{subfiles}

\begin{document}
\label{technologies}

\subsection{Technologie realizacji strony klienckiej}
Realizacja panelu administracyjnego oraz aplikacji klienckiej, przez które użytkownicy i administratorzy korzystają z systemu, zostały wykonane przy użyciu platformy programistycznej Angular (\url{https://angular.io/}).
Wykorzystaliśmy ją z powodu znajomości technologii, co umożliwiło sprawną implementację aplikacji.
W każdym przypadku skorzystaliśmy z języka TypeScript (\url{https://www.typescriptlang.org/}), który jest wykorzystywany przez Angulara oraz kompiluje się do JavaScriptu (\url{https://www.ecma-international.org/publications-and-standards/standards/ecma-262/}).

Aplikacja kliencka uruchamiana jest na systemie operacyjnym użytkownika poprzez technologię Electron (\url{https://www.electronjs.org/}), która umożliwia wyświetlanie aplikacji internetowej w postaci okna.
W wielkim uproszczeniu Electron korzysta z silnika Chromium, aby uruchomić stronę internetową jako aplikację i wyświetlić ją jak inne okna w systemie.
Ułatwiło nam to stworzenie aplikacji od razu na wiele systemów operacyjnych.
Dodatkowym plusem korzystania z Electrona jest możliwość odwołania się do silnika NodeJS (\url{https://nodejs.org/en/}), aby zwiększyć możliwości dostępu do zasobów systemu operacyjnego.
Wykorzystaliśmy go do uruchamiania podprocesu klienta RDP, aby po uzyskaniu sesji od razu zestawić połączenie z wybraną maszyną wirtualną.

Aby ułatwić implementację zaprojektowanego API (rozdział \ref{communication:api})) do komunikacji pomiędzy aplikacjami zewnętrznymi oraz nadzorcą, skorzystaliśmy z generatora kodu OpenAPI Generator (\url{https://openapi-generator.tech/}).
Dzięki niemu generowaliśmy gotowe szablony klas do komunikacji ze strony klienta oraz nadzorcy.
Zminimalizowało to błędy związane z błędną implementacją opisanego API.

\subsection{Technologie realizacji strony serwerowej}
W przypadku aplikacji nadzorcy oraz serwera wirtualizacji zdecydowaliśmy się na środowisko uruchomieniowe .NET 5.0 (\url{https://dotnet.microsoft.com/en-us/}) z językiem \texttt{C\#} (\url{https://docs.microsoft.com/en-us/dotnet/csharp/})).
Ekosystem .NET jest przez nas dobrze poznany i daje nam możliwości, aby ułatwić oraz przyśpieszyć naszą pracę.
Głównym ułatwieniem była duża baza bibliotek zgromadzona w menadżerze pakietów Nuget (\url{https://www.nuget.org/}), do którego dodaliśmy też własne pakiety, aby wydzielić wspólną część kodu aplikacji serwerowych.

Do zarządzania i monitorowania maszyn wirtualnych skorzystaliśmy z libvirta (\url{https://libvirt.org/}).
W serwerze wirtualizacji wykorzystaliśmy rozszerzoną przez nas bibliotekę odwołującą się do oryginalnych bibliotek libvirta.
Jest to technologia powszechnie znana i dobrze przez nas poznana. Był to najprostszy sposób do realizacji zarządcy maszyn wirtualnych.
Dodatkowo współpracuje ona także z technologią Vagrant (\url{https://www.vagrantup.com/}), która umożliwia proste powielanie maszyn wirtualnych z wzorca dostarczonego przez użytkownika.
Ułatwi ona administratorom tworzenie obrazów uruchamianych systemów operacyjnych.

Serwer wirtualizacji komunikuje się z Vagrantem poprzez specjalnie przygotowany plik wykonywalny z kodem w Ruby (\url{https://www.ruby-lang.org/en/}).
Vagrant wykonywany jest jako podproces serwera wirtualizacji w postaci poleceń powłoki.
Parametry przekazywane są poprzez zmienne środowiskowe ustawiane przez serwer wirtualizacji chwile przed uruchomieniem podprocesu.
Do dodatkowych konfiguracji maszyn wirtualnych użyliśmy technologii Ansible (\url{https://www.ansible.com/}).
Wybraliśmy ją, ponieważ jest znacznie mniej bezawaryjna niż konfigurowanie systemu przez skrypty powłoki.
Skorzystaliśmy z niej również do konfiguracji komputerów przed uruchomieniem pisanego przez nas systemu.

Do komunikacji pomiędzy modułami po stronie serwerowej skorzystaliśmy z asynchronicznego brokera wiadomości RabbitMQ (\url{https://www.rabbitmq.com/}).
Umożliwił nam on bezproblemową implementacje komunikacji pomiędzy modułami nadzorcy i serwera wirtualizacji.

\subsection{Technologie testowania}
Do testów jednostkowych aplikacji od strony klienckiej skorzystaliśmy z platformy Jest (\url{https://jestjs.io/}).
Ułatwia ona pisanie testów przy niepełnym obrazie systemu. Przy rozbudowanej komunikacji w naszym systemie było to bardzo przydatne.
W przypadku aplikacji serwerowych skorzystaliśmy z platformy NUnit (\url{https://nunit.org/}).
Znaliśmy ją z wcześniejszych projektów, więc mogliśmy od razu przystąpić do pracy.
Do testów komunikacji poprzez brokera wiadomości oraz integracji z libvirtem także skorzystaliśmy z NUnita.

Do testów integracyjnych aplikacji Angularowych korzystaliśmy z platformy Cypress (\url{https://www.cypress.io/}).
Zdecydowaliśmy się na nią, ponieważ wykonuje ona zrzut widoku podczas każdego kroku testów.
Wcześniej znane nam rozwiązania nie posiadały takich funkcji, przez co bardzo trudno było szukać problemów w testach.
Dodatkowe testy API wystawianego przez aplikacje nadzorcy wykonaliśmy Postmanem (\url{https://www.postman.com/}).

\subsection{Utrzymanie kodu}
Do sprawnego zarządzania projektem wykorzystaliśmy strukturę wielu repozytoriów dostępnych na stronie github.
Wspólne części kodu wydzielaliśmy w postaci pakietów NPM (\url{https://www.npmjs.com/}) oraz Nuget.
Aby zautomatyzować proces publikacji pakietów skorzystaliśmy ze środowiska ciągłej integracji Github Actions (\url{https://github.com/features/actions}).
Umożliwiło to praktycznie bezproblemową realizację komunikacji pomiędzy nadzorcą, a aplikacją kliencką i panelem administratora.
Tworzyliśmy automatycznie generowane pakiety z częściową implementacja obiektów transportowych oraz struktury zapytań.

\end{document}