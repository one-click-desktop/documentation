\documentclass[../opis-rozwiazania.tex]{subfiles}

\begin{document}

\subsection{Ansible}
Ansible został wykorzystany w~systemie do zaaplikowania zmiennej konfiguracji do każdej uruchamianej maszyny wirtualnej.
Podstawowo playbook może zawierać informacje o:
\begin{itemize}
  \item danych dostępowych do dysku sieciowego oraz wykorzystanym protokole,
  \item danych dostępowych do usługi katalogowej.
\end{itemize}
Playbook można rozszerzać o~potrzebne dane zależne od użycia.

\subsection{Vagrant}
Vagrant został wykorzystany w~celu łatwej parametryzacji oraz powtarzalnego tworzenia maszyn wirtualnych z~przygotowanego wcześniej obrazu systemu.

Wykorzystywany jest głównie mechanizm Vagrant Boxów, które są obrazami wcześniej przygotowanego systemu operacyjnego.
Aby system działał prawidłowo obraz systemu zamknięty w~Vagrant Boxie musi spełniać następujące warunki:
\begin{itemize}
  \item Użytkownicy muszą być pobierani z~usługi katalogowej.
  \item Katalogi domowe użytkowników muszą być na dysku sieciowym.
  \item Musi istnieć serwer RDP
\end{itemize}

\subsection{Libvirt z~QEMU}
Libvirt połączony z~QEMU jest wykorzystany do zarządzania maszynami wirtualnymi uruchamianymi na serwerze wirtualizacji.
Umożliwia on:
\begin{itemize}
  \item Tworzenie maszyn wirtualnych.
  \item Uruchamianie maszyn wirtualnych.
  \item Przyporządkowanie zasobów maszynom wirtualnym (w tym krat graficznych).
  \item Wyłączanie maszyn wirtualnych.
  \item Sprawdzanie, czy maszyna o~danej nazwie już działa.
\end{itemize}
\end{document}