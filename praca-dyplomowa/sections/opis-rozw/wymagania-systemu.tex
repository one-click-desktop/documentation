\documentclass[../opis-rozwiazania.tex]{subfiles}

\begin{document}
\label{system_requirements}
	
System do poprawnej pracy wymaga konfiguracji środowiska na wielu płaszczyznach.
	
\subsection{Komunikacja sieciowa między modułami}
System do poprawnego działania musi składać się z:
\begin{enumerate}
	\item Przynajmniej jednego nadzorcy.
	\item Serwerów wirtualizacji (może być ich 0).
	\item Maszyn wirtualnych uruchamnych na serwerach wirtualizacji.
	\item Serwera HTTP udostępniającego panel administracyjny.
	\item Brokera wiadomości do komunikacji wewnętrznej.
	\item Brokera wiadomości do komunikacje zewnętrznej (może być tym samym brokerem co wewnętrzny).
	\item Dowolnej liczby aplikacji klienckich.
\end{enumerate}

\subsubsection{Dostępność wewnętrznego brokera}
Broker wewnętrzny powinien być dostępny dla każdego nadzorcy oraz każdego serwera wirtualizacji.
Jest to wymagane do prawidłowej komunikacji pomiędzy nadzorcami a serwerami wirtualizacji.

\subsubsection{Dostępność zewnętrznego brokera}
Broker zewnętrzny powinien być dostępny dla każdego serwera wirtualizacji oraz dla każdego klienta łączącego się z systemem.
Jest to wymagane do stwierdzenia czy użytkownik nadal jest podłączony do maszyny wirtualnej. 

\subsubsection{Dostępność nadzorców}
Nadzorcy powinni być dostępni dla aplikacji klienckich, którzy poprzez nich komunikują się z resztą systemu. 
Administrator po pobraniu panelu administracyjnego z serwera HTTP powinien móc wysyłać zapytania do nadzorców.

\subsubsection{Dostępność serwerów wirtualizacji}
Serwery wirtualizacji nie muszą być dostępne dla żadnego z modułów.

\subsubsection{Dostępność serwera http z panelem administracyjnym}
Serwer HTTP powinien być dostępny dla każdego z administratorów.

\subsubsection{Dostępność maszyn wirtualnych}
Maszyny wirtualne powinny być dostępne dla każdego z klientów.
Jest to potrzebne do pracy na nich poprzez protokół RDP.

\subsection{Wymagania aplikacji klienckiej}
Dla systemu Windows aplikacja kliencka jako plik wykonywalny pobrana z \href{https://github.com/one-click-desktop/client/releases}{oficjalnych wydań} powinna uruchomić się bez żadnych wcześniejszych konfiguracji. Do uruchomienia wymagany jest jednak plik konfiguracji użytkownika dostępny w oficjalnych wydaniach.
Do działania korzysta ona z klienta RDP dostarczonego przez firmę Microsoft wraz z systemem Windows.
Aplikacja była testowana dla systemu Windows 10.
Dla starszych wersji systemu Windows aplikacja może nie działać prawidłowo.

W przypadku systemu Linux należy posiadać środowisko graficzne oraz mieć zainstalowanego klienta \href{https://www.freerdp.com/}{FreeRDP}.
Pozostałe zależności są dostarczone wewnątrz pliku \texttt{.appimage}.
Działanie aplikacji było testowane na dystrybucjach bazowanych na Debianie oraz ArchLinuxie.
Dla innych dystrybucji aplikacja może nie działać prawidłowo.

\subsection {Wymagania aplikacji nadzorcy i serwera panelu administracyjnego}
Do uruchomienia aplikacji nadzorcy i serwera HTTP z panelem administracyjnym potrzeba zainstalowanego \texttt{Dockera} w systemie.
Każdy z tych dwóch modułów można zbudować do kontenera, w którym wszystkie zależności zostaną spełnione.

\subsection{Wymagania serwera wirtualizacji}
\label{system_requirements.virtsrv_rquirements}
Serwer wirtualizacji oprócz działającej usługi \texttt{Dockera} potrzebuje dodatkowych usług.
Do prawidłowego działania wymagana jest działająca usługa zarządcy wirtualnych maszyn \texttt{libvirt}.
Aby prawidłowo uruchomić maszynę wirtualną potrzebna jest uruchomiona usługa zapory sieciowej oraz pakiet \texttt{dnsmasq}.
Wymagana jest także inicjalizacja struktur usługi \texttt{vagrant} dla użytkownika uruchamiającego serwer wirtualizacji.
Chodzi konkretnie o utworzenie folderu \texttt{.vagrant.d} w katalogu domowym użytkownika.

Aby uruchamiane maszyny wirtualne były dostępne dla innych urządzeń potrzeba utworzyć interfejs sieciowy w trybie bridge.
Nazwa interfejsu powinna być przekazana do pliku konfiguracyjnego serwera wirtualizacji.
Maszyny wirtualne uzyskają wtedy dostęp do sieci w taki sposób, jakby były fizycznymi komputerami w sieci.

W niektórych przypadkach podczas uruchomiania maszyny wirtualnej przy użyciu \texttt{vagranta} oraz uruchomiania wybranych kontenerów w \texttt{dockerze} komunikacja sieciowa z maszyną wirtualną poprzez podłączony interfejs w trybie bridge może zostać ograniczona. Należy wtedy dopilnować by w trakcie działanie systemu zapora sieciowa posiadała zasadę na samej górze łańcucha \texttt{FORWARD}, która będzie bezwarunkowo zezwalać trasować pakiety z urządzeń w trybie bridge.

\subsection{Automatyzacja konfiguracji}
\label{system_requirements.ansible_conf}
Przykładowa konfiguracja oraz narzędzia do automatycznej konfiguracji przed uruchomieniem dostępne sa w module \href{https://github.com/one-click-desktop/configuration}{\texttt{configuration}}.

Składa się on ze skryptów konfiguracyjnych Ansible, nazywanych dalej playbook, oraz zmiennych opisujących konfigurowane komputery.

By uruchomić skrypt dla pewnego systemu operacyjnego, który chcemy skonfigurować, musi on spełniać wymogi opisane w \href{https://docs.ansible.com/ansible/latest/plugins/connection.html#connection-plugins}{dokumentacji Ansible}.

\subsubsection{Grupy i zmienne}
Konfiguracja podzielona jest na 2 grupy: \texttt{overseer} i \texttt{virtsrv}.
Odpowiadają one za reprezentację systemów przygotowanych odpowiednio dla nadzorców oraz serwerów wirtualizacji.
Dodatkowo wytyczona jest sztuczna grupa \texttt{all} opisująca wszystkie konfigurowane systemy.

Jedyną wspólną zmienną dla wszystkich maszyn jest nazwa użytkownika, który będzie uruchamiał i odpowiadał za zasoby systemu OneClickDesktop.

Dla każdej maszyny z osobna należy zdefiniować dane dostępowe do komunikacji z konfigurowanym systemem. Dodatkowo należy podać hasło umożliwiające dostęp do uprawnień superużytkownika.

\subsubsection{Zmienne dla serwera wirtualizacji}
Playbook dla serwera wirtualizacji wykona wszystkie kroki opisane w \ref{system_requirements.virtsrv_rquirements}.
Aby tego dokonać należy zdefiniować dane dla tworzonego interfejsu typu bridge.
Należy zdefiniować nazwę interfejsu sieciowego, który zostanie połączony do nowo tworzonego urządzenia bridge o nazwie zdefiniowanej w pliku.
Playbook korzysta z \href{https://networkmanager.dev/}{NetworkManagera} do zmiany konfiguracji, zatem trzeba podać nazwę \textit{connection} (jednostka logiczna w NetworkManagerze) skojarzonego z początkowo wybranym interfejsem sieciowym.

\subsubsection{Zmienne dla nadzorcy}
Aby uruchomić nadzorcę wystarczą wspólne wymagania dla wszystkich grup.

\subsection{Wymagania szablonu maszyny wirtualnej}
\label{system_requirements.vagrant_box}
Maszyna wirtualna uruchamiana w ramach systemu OneClickDesktop musi być w postaci Vagrant Boxa.
Przykładowy box został utworzony w czasie rozwoju systemu i dostępny jest w chmurze Vagrant pod nazwą \href{https://app.vagrantup.com/smogork/boxes/archlinux-rdp}{smogork/archlinux-rdp}.
Box używa dystrybucji Arch Linux oraz spełnia wszystkie wymagania aby zostać uruchomionym w ramach systemu.

Do przygotowania szablonu proponujemy aby skorzystać właśnie z tego obrazu.
Jeżeli jednak jest potrzebne jest utworzenie niestandardowego szablonu to musi on:
\begin{itemize}
	\item Spełniać minimalne wymagania opisane w \href{https://www.vagrantup.com/docs/boxes/base}{dokumentacji Vagranta}.
	\item Mieć zainstalowany pewien menadżer okienek. Przykładowy szablon zawiera menadżera okienek XFCE.
	\item Przy uruchomieniu udostępniać usługę zdalnego pulpitu RDP. Przykładowy szablon korzysta z implementacji \href{http://xrdp.org/}{xrdp}.
	\item Każdy nowy interfejs sieciowy powinien być tak skonfigurowany, aby uzyskać adres IP z usługi DHCP.
	\item Przy starcie systemu urchamiać usługę \texttt{qemu-guest-agent}.
\end{itemize}
Nie ma ograniczenia na system operacyjny uruchamiany wewnątrz systemu OneClickDesktop. Jedynie taki szablon musi spełniać powyższe wymagania.

Aby zbudować własny szablon należy skonsultować się z \href{https://github.com/vagrant-libvirt/vagrant-libvirt#create-box}{dokumentacją wtyczki vagrant-libvirt}.
	
\end{document}