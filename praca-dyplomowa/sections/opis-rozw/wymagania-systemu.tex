\documentclass[../opis-rozwiazania.tex]{subfiles}

\begin{document}
\label{system_requirements}

System do poprawnej pracy wymaga konfiguracji środowiska na wielu płaszczyznach. Poniższy rozdział zawiera informacje o~konfiguracji systemu wymaganej do uruchomienia każdego z~modułów.

\subsection{Komunikacja sieciowa między modułami}
System do poprawnego działania z~pełną funkcjonalnością musi składać się z:
\begin{itemize}
  \item przynajmniej jednego nadzorcy (*),
  \item serwerów wirtualizacji,
  \item maszyn wirtualnych uruchamianych na serwerach wirtualizacji,
  \item serwera HTTP udostępniającego panel administracyjny,
  \item brokera wiadomości do komunikacji wewnętrznej (*),
  \item brokera wiadomości do komunikacje zewnętrznej (może być tym samym brokerem co wewnętrzny),
  \item dowolnej liczby aplikacji klienckich,
  \item systemu katalogowego przechowującego użytkowników
  \item dysku sieciowego zawierającego katalogi domowe użytkowników
\end{itemize}
Elementy wymagane do poprawnego uruchomienia pozostałych modułów oznaczone zostały symbolem (*).

Poniżej opisane zostały wymagania osiągalności połączeń sieciowych dla poszczególnych modułów.

\subsubsection{Dostępność wewnętrznego brokera}
Broker wewnętrzny powinien być dostępny dla każdego nadzorcy oraz każdego serwera wirtualizacji.
Jest to wymagane do prawidłowej komunikacji pomiędzy nadzorcami i~serwerami wirtualizacji.

\subsubsection{Dostępność zewnętrznego brokera}
Broker zewnętrzny powinien być osiągalny przez każdy serwer wirtualizacji oraz każdego klienta łączącego się z~systemem.
Jest to wymagane do stwierdzenia, czy użytkownik nadal jest podłączony do maszyny wirtualnej.

\subsubsection{Dostępność nadzorców}
Nadzorcy powinni być widoczni dla aplikacji klienckich, które poprzez nich komunikują się z~resztą systemu.
Administrator podczas używania panelu administracyjnego z~poziomu serwera HTTP powinien móc wysyłać zapytania do nadzorców.

\subsubsection{Dostępność serwerów wirtualizacji}
Serwery wirtualizacji nie muszą być widoczne przez żaden z~innych modułów.

\subsubsection{Dostępność serwera http z~panelem administracyjnym}
Serwer HTTP powinien być dostępny dla każdego z~administratorów.

\subsubsection{Dostępność maszyn wirtualnych}
Maszyny wirtualne powinny być osiągalny przez każdego z~klientów.
Jest to potrzebne do pracy na nich poprzez protokół RDP.

\subsubsection{Dostępność zewnętrznych modułów}
System katalogowy powinien być dostępny dla każdego nadzorcy oraz maszyny wirtualnej w~celu autoryzacji.
Dysk sieciowy powinien być widoczny przez maszyny wirtualne, aby mogły używać katalogów domowych użytkowników.

\subsection{Wymagania aplikacji klienckiej}
Aplikacja kliencka wspiera system Windows 10 lub nowszy oraz GNU/Linux w~dystrybucjach opartych na 64 bitowych systemach Ubuntu 18.04+ oraz Arch Linux.

Dla systemu Windows aplikacja kliencka, jako plik wykonywalny pobrany ze strony z~oficjalnymi wydaniami \parencite{ocd-client-releases}, powinna uruchomić się bez żadnych wcześniejszych konfiguracji. Do uruchomienia wymagany jest jednak plik konfiguracji użytkownika dostępny w~oficjalnych wydaniach.
Do działania korzysta ona z~klienta RDP dostarczonego przez firmę Microsoft wraz z~systemem Windows.
Aplikacja była testowana dla systemu Windows 10.
Dla starszych wersji systemu Windows aplikacja może nie działać prawidłowo.

W~przypadku systemu Linux należy posiadać środowisko graficzne oraz mieć zainstalowanego klienta FreeRDP (\url{https://www.freerdp.com/}).
Pozostałe zależności są dostarczone wewnątrz pliku \texttt{.appimage}.
Działanie aplikacji było testowane na dystrybucjach opartych na Debianie oraz Arch Linuxie.
Dla innych dystrybucji aplikacja może nie działać prawidłowo.

\subsection{Wymagania aplikacji nadzorcy i~serwera panelu administracyjnego}

\subsubsection{Wymagania sprzętowe}
Aplikacja nadzorcy i~serwer panelu administracyjnego może być uruchomiony na dowolnym komputerze z~możliwości uruchomienia usługi Dockera.
System operacyjny, pod którym pracuje komputer, nie ma znaczenia.

\subsubsection{Wymagania systemowe}
Do uruchomienia aplikacji nadzorcy i~serwera HTTP z~panelem administracyjnym potrzeba zainstalowanego Dockera w~systemie.
Każdy z~tych dwóch modułów można zbudować do kontenera, w~którym wszystkie zależności zostaną spełnione.

Do ręcznego uruchomienia aplikacji nadzorcy, bez pośrednictwa Dockera, wymagany jest .NET 5. W~przypadku panelu administratora do kompilacji wymagany jest NodeJS w~wersji \texttt{16.13.2}. Zbudowany moduł można udostępnić za pomocą dowolnego serwera HTTP.

\subsection{Wymagania serwera wirtualizacji}
\label{system_requirements.virtsrv_rquirements}

\subsubsection{Wymagania sprzętowe}
Serwer wirtualizacji zadziała prawidłowo na komputerze działającym pod kontrolą systemu operacyjnego z~grupy GNU/Linux.
Maszyna musi być podłączona do sieci poprzez przewodowy interfejs sieciowy (jest to wymagane przez interfejs typu bridge).
Wydajnościowo system musi być w~stanie uruchomić wirtualne maszyny z~usługą Dockera jednocześnie.
Jednak zalecamy aby system był wyposażony przynajmniej w~dwurdzeniowy procesor, 4GB pamięci operacyjnej oraz przynajmniej 100GB przestrzeni dyskowej.
Jeżeli system spełnia takie wymagania, to powinien być w~stanie uruchomić maszynę wirtualną pod kontrolą systemu bez przekazanego urządzenia PCI.

W~razie chęci wykorzystania funkcjonalności PCI Passthrough \parencite{pci-passthrough}, aby przekazać urządzenia do maszyny wirtualnej komputer musi spełniać dodatkowe wymagania.
Po aktywowaniu modułu IOMMU (I/O Memory Management Unit) \parencite{amd-iommu} każde z~urządzeń, które chcemy przekazać, musi znajdować się w~innej grupie IOMMU.
Prawidłowa izolacja urządzeń przekazywanych do maszyny wirtualnej od pozostałych urządzeń uruchamianych na serwerze jest kluczowa z~punktu bezpieczeństwa.

To czy prawidłowa izolacja urządzeń jest możliwa zależy od używanej w~komputerze płyty głównej.
Większość chipsetów przeznaczonych do komputerów biurowych używa tych samych linii komunikacji dla wielu urządzeń.
Należy więc skorzystać z~płyt przeznaczonych do procesorów dla entuzjastów (np. linia AMD Threadripper z~chipsetami z~linii X) albo profesjonalnych (np. linia AMD Epyc).
Posiadają one wiele linii PCI, które zwykle nie są współdzielone między urządzeniami.

Alternatywnym wyborem jest modyfikacja jądra systemu i~dodanie ACS Override patch \parencite{acs-override}.
Spowoduje to rozdzielenie wspólnych grup IOMMU na oddzielne wirtualne grupy.
Jednakże z~tak skonfigurowaną maszyną istnieje niebezpieczeństwo wykonania wyłomu poza wirtualną maszynę oraz zdestabilizowania pracy systemu.
Odradzamy wykorzystywania takiego rozwiązania w~przypadku publicznego dostępu do systemu.
W~przypadku braku odpowiedniego sprzętu oraz ograniczenia dostępu do systemu można spróbować z~niego skorzystać.

\subsubsection{Wymagania systemowe}
Serwer wirtualizacji, oprócz działającej usługi Dockera, wymaga usługi zarządcy maszyn wirtualnych libvirt w~wersji \texttt{7.10+}. W~przypadku ręcznego uruchamiania aplikacji, bez pośrednictwa Dockera, wymagany jest Vagrant w~wersji \texttt{2.2.11+} oraz .NET 5.

Aby prawidłowo uruchomić maszynę wirtualną potrzebna jest uruchomiona usługa zapory sieciowej oraz pakiet \texttt{dnsmasq}.
Wymagane jest także utworzenie struktur usługi Vagrant dla użytkownika uruchamiającego serwer wirtualizacji.
Chodzi konkretnie o~utworzenie folderu \texttt{.vagrant.d} w~katalogu domowym użytkownika.

Aby uruchamiane maszyny wirtualne były dostępne dla innych urządzeń potrzebne jest utworzenie interfejsu sieciowego w~trybie bridge.
Nazwa interfejsu powinna być przekazana do pliku konfiguracyjnego serwera wirtualizacji.
Maszyny wirtualne uzyskają wtedy dostęp do sieci w~taki sposób, jakby były fizycznymi komputerami w~sieci.

W~niektórych przypadkach, podczas uruchomiania maszyny wirtualnej przy użyciu Vagranta z~działającym Dockerem, komunikacja sieciowa z~maszyną wirtualną poprzez podłączony interfejs w~trybie bridge może zostać ograniczona.
Należy wtedy dopilnować, by w~trakcie działania systemu zapora sieciowa posiadała zasadę na samej górze łańcucha \texttt{FORWARD}, która będzie bezwarunkowo zezwalać trasować pakiety z~urządzeń w~trybie bridge.
Przy uruchomieniu serwera wirtualizacji w~kontenerze dockerowym trzeba dopilnować aby z~sieci, w~której pracuje kontener, można było osiągnąć sieć konfiguracyjną Vagranta.
W~przeciwnym wypadku próba uruchomienia maszyny wirtualnej zakończy się z~błędem połączenia SSH.

\subsection{Automatyzacja konfiguracji}
\label{system_requirements.ansible_conf}
Przykładowa konfiguracja oraz narzędzia do automatycznej konfiguracji przed uruchomieniem dostępne są w~module \texttt{configuration} \parencite{ocd-configuration}.
Składa się on ze skryptów konfiguracyjnych Ansible, nazywanych dalej playbook, oraz zmiennych opisujących konfigurowane komputery.

By uruchomić skrypt dla pewnego systemu operacyjnego, który chcemy skonfigurować, musi on spełniać wymogi opisane w~dokumentacji Ansible \parencite{ansible-connection}.

\subsubsection{Grupy i~zmienne}
Konfiguracja podzielona jest na 2 grupy: \texttt{overseer} i~\texttt{virtsrv}.
Odpowiadają one za reprezentację systemów przygotowanych odpowiednio dla nadzorców oraz serwerów wirtualizacji.
Dodatkowo wytyczona jest sztuczna grupa \texttt{all} opisująca wszystkie konfigurowane systemy.

Jedyną wspólną zmienną dla wszystkich maszyn jest nazwa użytkownika, który będzie uruchamiał i~odpowiadał za zasoby systemu OneClickDesktop.

Dla każdej maszyny z~osobna należy zdefiniować dane dostępowe do komunikacji z~konfigurowanym systemem. Dodatkowo należy podać hasło umożliwiające dostęp do uprawnień superużytkownika.

\subsubsection{Zmienne dla serwera wirtualizacji}
Playbook dla serwera wirtualizacji wykona wszystkie kroki opisane w~rozdziale \ref{system_requirements.virtsrv_rquirements}.
Aby tego dokonać należy zdefiniować dane dla tworzonego interfejsu typu bridge.
Należy podać nazwę interfejsu sieciowego, który zostanie połączony do nowo tworzonego urządzenia bridge o~nazwie zawartej w~pliku.
Playbook korzysta z~NetworkManagera (\url{https://networkmanager.dev/}) do zmiany konfiguracji.

\subsubsection{Zmienne dla nadzorcy}
Aby uruchomić nadzorcę wystarczą wspólne wymagania dla wszystkich grup.

\subsection{Wymagania szablonu maszyny wirtualnej}
\label{system_requirements.vagrant_box}
Maszyna wirtualna uruchamiana w~ramach systemu OneClickDesktop musi być dostarczona w~postaci Vagrant Boxa.
Przykładowy box został utworzony w~czasie rozwoju systemu i~dostępny jest w~chmurze Vagrant pod nazwą \texttt{smogork/archlinux-rdp} \parencite{ocd-vbox}.
Box używa dystrybucji Arch Linux oraz spełnia wszystkie wymagania aby zostać uruchomionym w~ramach systemu.

Do przygotowania szablonu proponujemy, aby skorzystać właśnie z~tego obrazu.
Jeżeli jednak potrzebne jest utworzenie niestandardowego szablonu to musi on:
\begin{itemize}
  \item Spełniać minimalne wymagania opisane w~dokumentacji Vagranta \parencite{vagrant-basebox}.
  \item Mieć zainstalowany menadżer okienek. Proponowany menadżer okienek, zawarty w~przykładowym szablonie, to XFCE w~wersji \texttt{4.16}. (\url{https://xfce.org/})
  \item Przy uruchomieniu udostępniać usługę zdalnego pulpitu RDP. Przykładowy szablon korzysta z~implementacji xrdp w~wersji \texttt{0.9.17-1}
  \item Każdy nowy interfejs sieciowy powinien być tak skonfigurowany, aby uzyskać adres IP z~usługi DHCP.
  \item Przy starcie systemu uruchamiać usługę \texttt{qemu-guest-agent}.
  \item Autoryzować użytkowników za pomocą systemu katalogowego \parencite{ldap-wiki}. Dokładne dane do ustanowienia połączenia mogą być przekazane za pomocą playbooka.
  \item Montować katalogi domowe użytkowników. Może to być również rozwiązane za pomocą playbooka.
  \item W~przypadku chęci skorzystania z~mechanizmu PCI Passthrough maszyna wirtualna powinna być zainstalowana w~trybie UEFI.
\end{itemize}
Nie ma ograniczenia co do systemu operacyjnego uruchamianego wewnątrz systemu OneClickDesktop, tak długo jak jest on w~stanie spełnić powyższe wymagania.

Aby zbudować własny szablon należy skonsultować się z~dokumentacją wtyczki vagrant-libvirt \parencite{vlibvirt-box}.

\end{document}