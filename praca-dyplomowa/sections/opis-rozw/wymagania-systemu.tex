\documentclass[../opis-rozwiazania.tex]{subfiles}

\begin{document}
\label{system_requirements}
	
System do poprawnej pracy wymaga konfiguracji środowiska na wielu płaszczyznach.
	
\subsection{Komunikacja sieciowa między modułami}
System składa się przynajmniej z:
\begin{enumerate}
	\item Przynajmniej jednego nadzorcy.
	\item Serwerów wirtualizacji (może być ich 0).
	\item Maszyny wirtualne uruchamiane na serwerach wirtualizacji.
	\item Serwera HTTP udostępniającego panel administracyjny.
	\item Brokera wiadomości do komunikacji wewnętrznej.
	\item Brokera wiadomości do komunikacje zewnętrznej (może być tym samym brokerem co wewnętrzny).
	\item Dowolnej liczby aplikacji klienckich.
\end{enumerate}

\subsubsection{Dostępność wewnętrznego brokera}
Broker wewnętrzny powinien być dostępny dla każdego nadzorcy oraz każdego serwera wirtualizacji.
Jest to wymagane do prawidłowej komunikacji pomiędzy nadzorcami a serwerami wirtualizacji.

\subsubsection{Dostępność zewnętrznego brokera}
Broker zewnętrzny powinien być dostępny dla każdego serwera wirtualizacji oraz dla każdego klienta łączącego się z systemem.
Jest to wymagane do stwierdzenia czy użytkownik nadal jest podłączony do maszyny wirtualnej. 

\subsubsection{Dostępność nadzorców}
Nadzorcy powinni być dostępni dla aplikacji klienckich, którzy poprzez nich komunikują się z resztą systemu. 
Administrator po pobraniu panelu administracyjnego z serwera http powinien móc wysyłać zapytania do nadzorców.

\subsubsection{Dostępność serwerów wirtualizacji}
Serwery wirtualizacji nie muszą być dostępne dla żadnego z modułów.

\subsubsection{Dostępność serwera http z panelem administracyjnym}
Serwer HTTP powinien być dostępny dla każdego z administratorów.

\subsubsection{Dostępność maszyn wirtualnych}
Maszyny wirtualne powinny być dostępne dla każdego z klientów.
Jest to potrzebne do pracy na nich poprzez protokół RDP.

\subsection{Wymagania aplikacji klienckiej}
Dla systemu Windows aplikacja kliencka jako plik wykonywalny pobrana z \href{https://github.com/one-click-desktop/client/releases}{oficjalnych wydań} powinna uruchomić się bez żadnych wcześniejszych konfiguracji.
Do działania skorzysta ona z klienta RDP dostarczonego przez Microsoft wraz z systemem Windows.

W przypadku systemu Linux należy zainstalować klienta \href{https://www.freerdp.com/}{FreeRDP}. Pozostałe zależności są dostarczone wewnątrz pliku \texttt{.appimage}.

\subsection {Wymagania aplikacji nadzorcy i serwera panelu administracyjnego}
Do uruchomienia aplikacji nadzorcy i serwera HTTP z panelem administracyjnym potrzeba zainstalowanego \texttt{Dockera} w systemie.
Każdy z tych dwóch modułów można zbudować do kontenera, w którym wszystkie zależności zostaną spełnione.

\subsection{Wymagania serwera wirtualizacji}
\label{system_requirements.virtsrv_rquirements}
Serwer wirtualizacji oprócz działającej usługi \texttt{Dockera} potrzebuje dodatkowych usług.
Do prawidłowego działania wymagana jest działająca usługa zarządcy wirtualnych maszyn \texttt{libvirt}.
Aby prawidłowo uruchomić maszynę wirtualna potrzebna jest uruchomiona usługa zapory sieciowej oraz pakiet \texttt{dnsmasq}.
Wymagana jest także inicjalizacja struktur usługi \texttt{vagrant} dla użytkownika uruchamiającego serwer wirtualizacji.
Konkretnie chodzi o utworzenie folder \texttt{.vagrant.d} w folderze domowym użytkownika.

Aby uruchamiane maszyny wirtualne były dostępne dla innych urządzeń potrzeba utworzyć interfejs sieciowy w trybie bridge.
Nazwa interfejsu powinna być przekazana do pliku konfiguracyjnego serwera wirtualizacji.
Maszyny wirtualne uzyskają wtedy dostęp do sieci w sposób jakby były fizycznymi komputerami w sieci.

Czasami przy uruchomieniu maszyny wirtualnej przy użyciu \texttt{vagranta} oraz uruchomieniu wybranych kontenerów w \texttt{dockerze} komunikacja sieciowa z maszyny wirtualnej poprzez podłączony interfejs w trybie bridge może zostać ograniczona. Wtedy należy pilnować by w trakcie działanie systemu zapora sieciowa posiadała zasadę na samej górze łańcucha \texttt{FORWARD}, która będzie bezwarunkowo zezwalać trasować pakiety z urządzeń w trybie bridge. 

\subsection{Automatyzacja konfiguracji}
\label{system_requirements.ansible_conf}
Przykładowa konfiguracja oraz narzędzia do automatycznej konfiguracji przed uruchomieniem dostępne sa w module \href{https://github.com/one-click-desktop/configuration}{\texttt{configuration}}.

Składa się on ze skryptów konfiguracyjnych Ansible, nazywanych dalej playbook, oraz zmiennych opisujących konfigurowane komputery.

By uruchomić skrypt dla pewnego systemu operacyjnego, który chcemy skonfigurować, musi on spełniać wymogi opisane w \href{https://docs.ansible.com/ansible/latest/plugins/connection.html#connection-plugins}{dokumentacji Ansible}.

\subsubsection{Grupy i zmienne}
Konfiguracja podzielona jest na 2 grupy: \texttt{overseer} i \texttt{virtsrv}.
Odpowiadają one za reprezentację systemów przygotowanych odpowiednio dla nadzorców oraz serwerów wirtualizacji.
Dodatkowo wytyczona jest sztuczna grupa \texttt{all} opisująca wszystkie konfigurowane systemy.

Jedyną wspólną zmienną dla wszystkich maszyn jest nazwa użytkownika, który będzie uruchamiał i odpowiadał za zasoby systemu OneClickDesktop.

Dla każdej maszyny z osobna należy dane dostępowe do komunikacji z konfigurowanym systemem. Dodatkowo należy podać hasło dla dostępu do uprawnień superużytkownika.

\subsubsection{Zmienne dla serwera wirtualizacji}
Playbook dla serwera wirtualizacji wykona wszystkie kroki opisane w \ref{system_requirements.virtsrv_rquirements}.
Aby tego dokonać należy zdefiniować dane dla tworzonego interfejsu typu bridge.
	
\end{document}