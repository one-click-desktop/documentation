\documentclass[../analiza-rozwiazania.tex]{subfiles}

\begin{document}

Testy przeprowadzaliśmy na dwóch komputerach podłączonych do wspólnej sieci lokalnej.
Usługa OpenLDAP była uruchomiona na innym komputerze oraz udostępniała testową bazę użytkowników.
Każdy z nich pracował pod kontrolą systemu operacyjnego Arch Linux w wersji ze stycznia 2022 roku.
Obydwa pracowały z 8 GB pamięci operacyjnej oraz 4 rdzeniowym procesorem o 8 wątkach (Intel i7-2600K oraz i7-4790).
Do testów skonfigurowaliśmy serwery wirtualizacji, aby miały do dyspozycji 6 wątków oraz 4096 MB pamięci operacyjnej.

Niestety taka platforma uniemożliwiła przetestowanie funkcjonalności PCI Passthrough (patrz wymagania sprzętowe serwera wirtualizacji, rozdział \ref{system_requirements.virtsrv_rquirements}).
Budowa ich płyt głównych nie pozwalała na całkowita izolację urządzeń podłączonych do złączy PCI Express.
Skorzystaliśmy z innego komputera, który posiadał inną konstrukcję płyty głównej i każde z urządzeń zostało prawidłowo odizolowane dzięki przydzieleniu innych adresów pamięci.
Pracował on pod kontrola systemu operacyjnego Arch Linux ze stycznia 2022 roku.
Posiadał on 48GB pamięci operacyjnej oraz 12 rdzeniowy procesor o 24 wątkach (AMD Threadripper 1920X).
Uruchamiany na nim serwer wirtualizacji miał do dyspozycji 20 wątków oraz 43008 MB pamięci operacyjnej.

Na każdym z tych komputerów można było uruchomić aplikację serwera wirtualizacji, panelu administracyjnego oraz nadzorcy.
\end{document}