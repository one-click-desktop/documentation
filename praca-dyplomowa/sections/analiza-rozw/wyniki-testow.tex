\documentclass[../analiza-rozwiazania.tex]{subfiles}

\begin{document}
Po zmianie w~kodzie testy automatyczne często kończyły się błędem.
Był to dla nas znak, aby sprawdzić co się dzieje i~wprowadzić odpowiednie poprawki.

Próby wdrożenia systemu w~środowisku testowym przyniosły odkrycie wielu błędów logicznych w~naszych wstępnych pomysłach jak i~samej implementacji.
Przykładowo zorientowaliśmy się, że serwer wirtualizacji wyłącza się w~przypadku braku adresu IP do komunikacji z~maszyną wirtualną.
Mogliśmy zmienić serwer, aby ignorował taką sytuację, ale wtedy użytkownik nie mógłby się komunikować z~maszyną wirtualną.
Podjęliśmy ostatecznie decyzję o~wyłączeniu takiej maszyny, zgłoszeniu błędu do logów i~wycofaniu jej z~modelu.
Dodatkowo pomogło nam to też lepiej zrozumieć wymagania sprzętowe jak i~zależności naszego systemu.

Przy wykonywaniu testów akceptacyjnych mieliśmy szansę dopracować stabilność oraz ponownie skonfrontować wstępne pomysły z~rzeczywistością.
Po poprawkach błędów i~upewnieniu się, że wszystkie scenariusze akceptacyjne są już spełnione, system był już wystarczająco stabilny, aby można było z~niego korzystać.
Zrozumieliśmy wtedy lepiej jakie rozwiązania nie są wygodne i~wymagają poprawek albo nawet przeprojektowania.
Nasze przemyślenia o~nieudanych elementach systemu można znaleźć w~podsumowaniu (rozdział \ref{final_system_form}).
\end{document}