\documentclass[../analiza-rozwiazania.tex]{subfiles}

\begin{document}
Testy automatyczne często kończyły się błędem po zmianie w kodzie.
Był dla nas znak aby sprawdzić co się dzieje i wprowadzić odpowiednie poprawki.

Próby wdrożenia systemu w środowisku testowych przyniosły odkrycie wielu błędów logicznych w naszych wstępnych pomysłach jak i samej implementacji.
Dodatkowo pomogło nam to też lepiej zrozumieć wymagania sprzętowe jak i zależności naszego systemu.

Przy wykonywaniu testów akceptacyjnych mieliśmy szansę dopracować stabilność oraz ponownie skonfrontować wstępne pomysły z rzeczywistością.
Po poprawkach błędów i upewnieniu się, że wszystkie scenariusze akceptacyjne są już spełnione system był już wystarczająco stabilny aby można było z niego korzystać.
Jednak zrozumieliśmy wtedy lepiej jakie rozwiązania nie sa wygodne i wymagają poprawek albo nawet przeprojektowania.
Nasze wnioski o rozwiązaniach do poprawki można znaleźć w podsumowaniu (rozdział \ref{final_system_form}).
\end{document}