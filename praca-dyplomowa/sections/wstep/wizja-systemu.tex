\documentclass[../wstep.tex]{subfiles}

\begin{document}

Tworzony system ma za zadanie umożliwiać zdalną pracę za pomocą protokołu zdalnego pulpitu. System skierowany jest w stronę firm zatrudniających wielu pracowników, które chcą scentralizować sprzęt używany do pracy zdalnej.

Użytkownikami końcowym są pracownicy, którzy za pomocą okienkowej aplikacji klienckiej mogą uzyskać sesję do pracy zdalnej. Użytkownik podczas łączy się za pomocą protokołu zdalnego pulpitu z maszyną wirtualną uruchamiającą obraz systemu GNU/Linux. Uruchamianie i zarządzanie maszynami jest zadaniem aplikacji działającej na rzeczywistej maszynie, która udostępnia swoje zasoby maszynom wirtualnym. Aplikacja ta, oraz rzeczywista maszyna uruchamiająca ją, nazywana jest dalej serwerem wirtualizacji. Aplikacje te działają niezależnie od siebie i nie ma teoretycznego ograniczenia na ich liczbę w systemie. Komunikacją z użytkownikami oraz zarządzaniem systemem zajmuje się aplikacja nadzorcza. Ilość jej instancji również jest teoretycznie nieograniczona, co umożliwia balansowanie obciążeniem.

System pozwala na tworzenie maszyn wirtualnych różnych typów, czyli kombinacji zasobów systemowych udostępnianych dla maszyny wirtualnej, oraz faktu czy ma ona bezpośredni dostęp do karty graficznej maszyny, na której pracuje. Do używania systemu użytkownik musi posiadać konto w systemie katalogowym, który umożliwia użytkownikom dostęp do własnego folderu domowego na każdej maszynie. System katalogowy nie jest ujęty w obrębie systemu, ale jego poprawna konfiguracja jest wymagana do użytkowania systemu.

System udostępnia panel administracyjny w postaci strony WWW umożliwiający podgląd obciążenia i stanu systemu przez upoważnione osoby. Komunikacja aplikacji klienckiej z aplikacją nadzorczą oraz panel administratora wykorzystują komunikację za pomocą protokołu HTTP. Możliwe jest użycie szyfrowanego protokołu HTTPS, pod warunkiem użycia poprawnych certyfikatów SSL/TSL.

\end{document}