\documentclass[../wstep.tex]{subfiles}

\begin{document}

Tworzony system ma za zadanie umożliwiać zdalną pracę za pomocą protokołu zdalnego pulpitu. Rozwiązanie pozwala na wykorzystanie istniejących stacji roboczych do tworzenia wirtualnych stanowisk.

Użytkownikami końcowymi są pracownicy, którzy za pomocą okienkowej aplikacji klienckiej mogą uzyskać sesję do pracy zdalnej. Użytkownik podczas pracy łączy się za pomocą protokołu zdalnego pulpitu z maszyną wirtualną uruchamiającą obraz systemu GNU/Linux. Uruchamianie i zarządzanie maszynami jest zadaniem aplikacji działającej na rzeczywistej stacji roboczej, która udostępnia swoje zasoby maszynom wirtualnym. Aplikacja ta oraz rzeczywista maszyna uruchamiająca ją nazywana jest dalej serwerem wirtualizacji. Serwery działają niezależnie od siebie i nie ma teoretycznego ograniczenia ich liczby w systemie. Komunikacją z użytkownikami oraz zarządzaniem systemem zajmuje się aplikacja nadzorcza. Ilość jej instancji również jest teoretycznie nieograniczone co umożliwia balansowanie obciążeniem oraz zwiększa odporność systemu na awarie.

System pozwala na tworzenie maszyn wirtualnych różnych typów. Typ określa ilość zasobów systemowych udostępnianych maszynie wirtualnej oraz to, czy ma ona bezpośredni dostęp do karty graficznej maszyny, na której pracuje. Do używania systemu użytkownik musi posiadać konto w systemie katalogowym, na które loguje się podczas użytkowania systemu. Foldery domowe pracowników montowane są wewnątrz maszyny z zewnętrznego dysku sieciowego i umożliwiają przechowywanie danych między połączeniami.

System udostępnia panel administracyjny w postaci strony WWW umożliwiający podgląd obciążenia i stanu systemu przez upoważnione osoby. Applikacja kliencka oraz panel administratora komunikuje się z aplikacją nadzorczą wykorzystując protokół HTTP. Możliwe jest użycie szyfrowanego protokołu HTTPS \parencite{rfc2818}, pod warunkiem użycia poprawnych certyfikatów SSL/TSL.

\end{document}