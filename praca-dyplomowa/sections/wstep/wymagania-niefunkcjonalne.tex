\documentclass[../wstep.tex]{subfiles}

\begin{document}
\label{requirements:nonfunctional}

\begin{table}[H]
    \caption[Wymagania niefunkcjonalne]{Wymagania niefunkcjonalne}
    \label{non-functional}
    \centering
    \begin{tabular}{|p{0.2\textwidth}|p{0.2\textwidth}|p{0.6\textwidth}|}
        \hline Grupa wymagań                            & Nr wymagania & Opis                                                                                                                                                                                                                       \\ \hline
        \multirow[t]{10}{=}{Użytkowanie (Usability)}    & 1            & Aplikacja kliencka ma działać na systemach operacyjnych MS Windows (Windows 10) oraz GNU/Linux (Arch Linux). Aplikacja na systemach GNU/Linux wymaga zainstalowanego klienta RDP zgodnego z XRDP \parencite{xrdp-clients}. \\ \cline{2-3}
                                                        & 2            & Aplikacja kliencka musi udostępniać możliwość użycia własnego klienta RDP do nawiązania połączenia z maszyną wirtualną.                                                                                                     \\ \cline{2-3}
                                                        & 3            & Maszyny wirtualne muszą mieć dostęp do systemu przechowującego konta użytkowników wraz z ich katalogami domowymi.                                                                                                           \\ \hline
        \multirow[t]{7}{=}{Niezawodność (Reliability)}  & 4            & System musi być odporny na awarie poszczególnych serwerów wirtualizacji i kontynuować działanie w sposób niezauważalny dla użytkowników nie używających danego. serwera.                                                    \\ \cline{2-3}
                                                        & 5            & Awaria nadzorcy może spowodować uniemożliwienie rozpoczęcia nowych sesji, ale nie może przerwać istniejących sesji.                                                                                                         \\ \hline
        \multirow[t]{10}{=}{Wydajność (Performance)}    & 6            & Łącznie zużywane zasoby przez maszyny wirtualne na poszczególnym serwerze wirtualizacji nie mogą przekroczyć wcześniej zdefiniowanych. limitów                                                                              \\ \cline{2-3}
                                                        & 7            & Nadzorca musi balansować obciążeniem serwerów wirtualizacji.                                                                                                                                                                 \\ \cline{2-3}
                                                        & 8            & W systemie zawsze musi istnieć jedna działająca maszyna wirtualna nie połączona z żadną sesją, aby można było ją szybko przydzielić użytkownikowi.                                                                         \\ \cline{2-3}
                                                        & 9            & Zwolnione maszyny wirtualne, które nie są wykorzystywane jako zapas, muszą być wyłączane                                                                                                                                   \\ \hline
        \multirow[t]{3}{=}{Utrzymanie (Supportability)} & 10           & Możliwe jest działanie więcej niż jednego nadzorcy w systemie, w celu zwiększenia dostępności lub przeprowadzenia prac utrzymaniowych                                                                                      \\
        \hline
    \end{tabular}
\end{table}

\end{document}