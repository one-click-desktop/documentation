\documentclass[../wstep.tex]{subfiles}

\begin{document}

Jednym z bardziej znanych rozwiązań podobnych do naszego systemu jest system Citrix (\url{https://www.citrix.com/pl-pl/}).
Oferuje on szeroki wachlarz usług dostępu do pulpitów zdalnych, aplikacji oraz usług w chmurze przeznaczonych do pracy z dowolnego miejsca w sieci.
Jedna z usług jest bardzo podobna do naszego systemu - \textit{Citrix Virtual Apps and Desktops} \parencite{citrix-daas}.
Przy kupnie systemu można wybrać czy interesuje nas udostępnianie całego pulpitu zdalnego jako serwis (tzw. DaaS - Desktop as a Service), czy zdalny dostęp do wybranego komputera.

System ten posiada także zbiór komputerów łączonych w klaster oraz aplikacje balansującą, która równomiernie rozkłada zużycie wszystkich komputerów w klastrze.
Możliwe jest skorzystanie z gotowej infrastruktury zaoferowanej przez Citrixa albo wybudowanie własnego klastra.

Jedną z ważniejszych różnic Citrixa od naszego systemu jest większa wirtualizacja zasobów podłączonych do maszyny wirtualnej.
W naszym systemie możliwe jest bardziej statyczne przyporządkowanie zasobów (np. karta graficzna) co pozwala na uzyskanie lepszej wydajności do nietypowych zastosowań biurowych (np. symulacje numeryczne w programach typu CAD).
Nasz system może przekazać do maszyny wirtualnej dowolne urządzenie PCI zdefiniowane w konfiguracji.
Umożliwia to prace z bardzo nietypowymi urządzeniami, nawet zdalnie.

Dodatkowo nasz system pozwala łatwo wykorzystać istniejące komputery do stworzenia klastra.
Konfiguracja każdego komputera w klastrze może być ustawiona oddzielnie. System odpowiednio zarządzi różnymi typami maszyn.
Może to obniżyć koszty, jeżeli pracownicy często zmieniają swój tryb pracy.

\end{document}