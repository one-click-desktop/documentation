\documentclass[../wstep.tex]{subfiles}

\begin{document}
	
Jednym z bardziej znanych rozwiązań podobnych do naszego systemu jest system Citrix(\url{https://www.citrix.com/pl-pl/}).
Oferuje on szeroki wachlarz usług dostępu do pulpitów zdalnych, aplikacji, usług w chmurze przeznaczony do pracy z dowolnego miejsca w sieci.
%TODO: przypis
Jedna z usług jest bardzo podobną do naszego systemu - Citrix Virtual Apps and Desktops\textit{Dodac przypis!}(https://www.citrix.com/pl-pl/products/citrix-virtual-apps-and-desktops/).
Przy kupnie systemu można wybrać czy interesuje nas udostępnianie całego pulpitu zdalnego jako serwis (tzw. DaaS) albo po prostu zdalny dostęp do wybranego komputera.

System ten także posiada zbiór komputerów łączony w klaster oraz aplikacje balansującą, który równomiernie rozkłada zużycie wszystkich komputerów w klastrze.
Możliwe jest w skorzystanie z gotowej infrastruktury zaoferowanej przez Citrixa albo wybudowanie własnego klastra. 

Jedną z ważniejszych różnic Citrixa od naszego systemu jest większa wirtualizacja zasobów podłączonych do maszyny wirtualnej.
W naszym systemie możliwe jest bardziej statyczne przyporządkowanie zasobów (np. karta graficzna), co pozwala na uzyskanie lepszej wydajności do nie typowych zastosowań biurowych (np. symulacje numeryczne w programach typu CAD).
Dodatkowo nasz system pozwala łatwo wykorzystać istniejące komputery do stworzenia klastra.
Konfiguracja każdego komputera w klastrze może być ustawiona oddzielnie, aplikacja nadzorcy odpowiednio zarządzi różnymi typami maszyn.
Taki fakt może obniżyć koszty, jeżeli pracownicy często zmieniają swój tryb pracy.
	
\end{document}
