\documentclass[../wstep.tex]{subfiles}

\begin{document}

\begin{itemize}
    \item Aplikacja kliencka - aplikacja okienkowa uruchamiana na komputerze użytkownika, która umożliwi komunikację z systemem oraz uruchomienie zewnętrznego programu implementującego protokół RDP.
    \item Aplikacja nadzorcza (Nadzorca) - aplikacja, która przetwarza zapytania od aplikacji klienckiej oraz komunikuje się ze wszystkimi serwerami wirtualizacji. Na podstawie tych informacji buduje model zajętości każdego z serwerów wirtualizacji oraz decyduje kiedy, i na którym serwerze, trzeba uruchomić nowe maszyny wirtualne. Decyduje również, do której wirtualnej maszyny ma podłączyć się użytkownik proszący o utworzenie sesji.
    \item Serwer wirtualizacji - komputer, który udostępnia swoje zasoby (rdzenie procesora, karty graficzne, pamięć RAM oraz przestrzeń dyskową) w postaci uruchamianych na nim maszyn wirtualnych. Komputer ten uruchamia aplikację, która odpowiada na zapytania aplikacji nadzorczej oraz wykonuje operacje na maszynach wirtualnych (uruchamianie i wyłączanie). Komputer może uruchamiać co najwyżej jedną aplikację, dlatego zarówno komputer, jak i aplikację, nazywamy serwerem wirtualizacji.
    \item Maszyna wirtualna CPU - maszyna systemowa emulująca, lub para-emulująca, sprzęt i służąca do uruchamiania systemu operacyjnego. Udostępnia użytkownikowi podstawowe zasoby (procesor, pamięć RAM i przestrzeń dyskowa). Uruchamiana jest na serwerze wirtualizacji z liczbą zasobów określoną w konfiguracji. Maszyna wirtualna uruchamia system operacyjny GNU/Linux (ArchLinux).
    \item Maszyna wirtualna GPU - maszyna analogiczna do maszyny wirtualnej CPU. Wyróżnia się przekazaną na wyłączność, za pośrednictwem mechanizmu GPU Passthrough, kartą graficzną podłączona do serwera wirtualizacji.
    \item RDP - protokół zdalnego dostępu do pulpitu od firmy Microsoft \parencite{rdp}. Maszyny wirtualne uruchamiają serwer RDP(xrdp - \url{http://xrdp.org/}), który umożliwia zdalną pracę za pośrednictwem protokołu RDP.
    \item Sesja - jednorazowy dostępu użytkownika do systemu oraz maszyny wirtualnej. Utworzenie sesji wiąże się z przypisaniem do użytkownika konkretnej maszyny wirtualnej, na której będzie pracować. Sesja kończy się w przypadku, gdy użytkownik poinformuje system o zakończeniu pracy lub gdy minie czas oczekiwania na odzyskanie połączenia jego utracie.
    \item Vagrant Box \parencite{vagrantbox} - przygotowany wcześniej obraz maszyny wirtualnej, który umożliwia zmianę dostępnych zasoby. Uruchamiają się bardzo powtarzalnie w środowisku programu Vagrant. Obrazy te używane są do tworzenia maszyn wirtualnych.
    \item Ansible playbook \parencite{ansible-playbook} - skrypt konfiguracyjny dla systemu operacyjnego, który umożliwia parametryzację oraz wykonywanie podczas uruchamiania Vagrant Boxa.
    \item Panel administratora - aplikacja przeglądarkowa, która umożliwia administratorowi systemu podgląd listy serwerów wirtualizacji znajdujących się w systemie oraz zajętości zasobów.
    \item Konto użytkownika - profil użytkownika w systemie, do którego ma dostęp na każdej maszynie wirtualnej. Używając przygotowanych wcześniej danych logowania może za ich pomocą logować się do maszyn wirtualnych. Przechowywane są w zewnętrznym (poza opisanym systemem) systemie katalogowym.
    \item Katalog użytkownika - prywatny folder dostępny dla użytkownika na każdej maszynie wirtualnej. Przechowywany na zewnętrznym (poza opisanym systemem) dysku sieciowym.
    \item Konfiguracja stała - konfiguracja maszyny wirtualnej, która nie zmienia się w zależności od miejsca uruchomienia. Docelowo ta konfiguracja ma być zapisana w Vagrant Boxie. W razie potrzeby można ja także zdefiniować w odpowiednim Ansible playbooku.
    \item Konfiguracja zmienna - konfiguracja maszyny wirtualnej, która zmienia się w zależności od miejsca uruchomienia. Jest definiowana w odpowiednim Ansible playbooku uruchamianym przy każdym włączeniu maszyny.
\end{itemize}

\end{document}