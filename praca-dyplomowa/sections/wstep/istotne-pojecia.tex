\documentclass[../wstep.tex]{subfiles}

\begin{document}
\begin{itemize}
  \item \textbf{Aplikacja kliencka} - aplikacja okienkowa uruchamiana na komputerze użytkownika, która umożliwia komunikację z~systemem oraz uruchomienie zewnętrznego programu implementującego protokół RDP. Działa pod systemem operacyjnym Windows 10 oraz GNU/Linux.
  \item \textbf{Panel administratora} - aplikacja przeglądarkowa, która umożliwia administratorowi systemu podgląd statusu serwerów wirtualizacji znajdujących się w~systemie oraz zajętość zasobów.
  \item \textbf{Aplikacja nadzorcza (Nadzorca)} - aplikacja, która przetwarza zapytania aplikacji klienckiej lub panelu administratora oraz komunikuje się ze wszystkimi serwerami wirtualizacji. Na podstawie tych informacji buduje model zajętości każdego z~serwerów wirtualizacji oraz decyduje kiedy, i~na którym serwerze, trzeba uruchomić nowe maszyny wirtualne. Decyduje również, do której wirtualnej maszyny ma podłączyć się użytkownik proszący o~utworzenie sesji. Aplikacja może działać na dowolnym komputerze spełniającym wymagania.
  \item \textbf{Serwer wirtualizacji} - komputer, który udostępnia swoje zasoby (rdzenie procesora, karty graficzne, pamięć RAM oraz przestrzeń dyskową) w~postaci uruchamianych na nim maszyn wirtualnych. Komputer ten uruchamia aplikację, która odpowiada na zapytania aplikacji nadzorczej oraz wykonuje operacje na maszynach wirtualnych (uruchamianie i~wyłączanie). Komputer może uruchamiać co najwyżej jedną aplikację, dlatego zarówno maszynę, jak i~aplikację, nazywamy serwerem wirtualizacji. Program uruchamiany jest na stacji roboczej, która ma być częścią klastra.
  \item \textbf{Broker wiadomości} - aplikacja przekazująca wiadomości pomiędzy połączonymi z~nią innymi aplikacjami.
  \item \textbf{Maszyna wirtualna CPU} - maszyna systemowa emulująca lub para-emulująca sprzęt i~służąca do uruchamiania systemu operacyjnego. Udostępnia użytkownikowi podstawowe zasoby (procesor, pamięć RAM i~przestrzeń dyskowa). Uruchamiana jest na serwerze wirtualizacji z~liczbą zasobów określoną w~konfiguracji. Maszyna wirtualna uruchamia system operacyjny GNU/Linux (ArchLinux - \url{https://archlinux.org/}).
  \item \textbf{Maszyna wirtualna GPU} - maszyna analogiczna do maszyny wirtualnej CPU. Wyróżnia się przekazaną na wyłączność, za pośrednictwem mechanizmu PCI Passthrough, kartą graficzną podłączoną do serwera wirtualizacji.
  \item \textbf{RDP} - protokół zdalnego dostępu do pulpitu od firmy Microsoft \parencite{rdp}. Maszyny wirtualne uruchamiają serwer RDP (xrdp - \url{http://xrdp.org/}), który umożliwia pracę za pośrednictwem protokołu zdalnego dostępu.
  \item \textbf{Sesja} - jednorazowy dostępu użytkownika do systemu oraz maszyny wirtualnej. Utworzenie sesji wiąże się z~przypisaniem do użytkownika konkretnej maszyny wirtualnej, na której będzie pracować. Sesja kończy się w~przypadku, gdy przez pewien czas od opuszczenia maszyny lub utraty połączenia z~systemem użytkownik nie połączy się ponownie.
  \item \textbf{Vagrant Box \parencite{vagrantbox}} - przygotowany wcześniej obraz maszyny wirtualnej, który umożliwia zmianę dostępnych zasobów. Uruchamiają się z~obrazu pierwotnego w~środowisku programu Vagrant. Obrazy te używane są do tworzenia maszyn wirtualnych.
  \item \textbf{Ansible playbook \parencite{ansible-playbook}} - skrypt konfiguracyjny dla systemu operacyjnego, który umożliwia parametryzację oraz wykonywanie określonych akcji po uruchomieniu Vagrant Boxa.
  
  \item \textbf{Konto użytkownika} - profil użytkownika w~systemie, do którego ma dostęp na każdej maszynie wirtualnej. Używane do logowania się do systemu oraz maszyn wirtualnych. Przechowywane jest w~zewnętrznym systemie katalogowym.
  \item \textbf{Katalog użytkownika} - prywatny folder dostępny dla użytkownika na każdej maszynie wirtualnej. Przechowywany na zewnętrznym dysku sieciowym oraz montowane na maszynach wirtualnych.
  \item \textbf{Konfiguracja stała} - konfiguracja maszyny wirtualnej, która nie zmienia się w~zależności od miejsca uruchomienia. Zapisana jest w~Vagrant Boxie. W~razie potrzeby można ją także zdefiniować w~odpowiednim Ansible playbooku.
  \item \textbf{Konfiguracja zmienna} - konfiguracja maszyny wirtualnej, która zmienia się w~zależności od miejsca uruchomienia. Jest definiowana w~odpowiednim Ansible playbooku uruchamianym przy każdym włączeniu maszyny.
\end{itemize}

\end{document}