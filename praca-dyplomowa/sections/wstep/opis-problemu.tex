\documentclass[../wstep.tex]{subfiles}

\begin{document}

Aktualnie można zaobserwować dużą zmianę na rynku pracy.
Z~powodu globalnej epidemii wiele firm zdecydowało się na zmianę trybu pracy ze stacjonarnego na zdalny. W~2020 roku około $70\%$ pracowników zatrudnionych na pełen etat w~Stanach Zjednoczonych pracowało z~domu \parencite{remote-2020}.
Nawet po złagodzeniu obostrzeń znaczna część miejsc pracy pozostała przy takim trybie lub przyjęło hybrydową formę pracy. Około $89\%$ europejskich firm planuje pozostać przy hybrydowym trybie pracy nawet po zakończeniu epidemii \parencite{remote-2021}.
Przy pracy zdalnej pracownicy często muszą łączyć się za pomocą funkcji zdalnego pulpitu z~komputerami znajdującymi się w~biurze.
Może to wynikać z~niewystarczającej wydajności sprzętu pracownika lub konieczności dostępu do specyficznych programów albo zasobów.
W~takich sytuacjach komputer, z~którym łączy się użytkownik musi być uruchomiony, a~w~przypadku awarii - zrestartowany.
Dodatkowo taki dostęp może być wymagany jedynie przez ograniczony czas, co powoduje, że dużą część czasu maszyna pozostaje włączona, ale nieużywana.

Możliwym sposobem na złagodzenie tego problemu jest użycie zmniejszonej liczby stacji roboczych, które mogą być używane przez większą liczbę pracowników jednocześnie za pośrednictwem maszyn wirtualnych.
Zmniejsza to liczbę działających maszyn fizycznych, a~zarządzanie wirtualnymi może być wykonywane zdalnie.

Tworzony przez nas system umożliwi zmniejszenie liczby stanowisk na rzecz jednego \textit{przezroczystego komputera}.
Pracownicy będą mogli poprosić o~zdalny dostęp do maszyny z~zasobami systemowymi dopasowanymi do ich potrzeb.
System sam zadba o~udostępnienie i~zarządzanie tymi zasobami.

\end{document}