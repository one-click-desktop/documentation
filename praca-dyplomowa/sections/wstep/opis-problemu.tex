\documentclass[../wstep.tex]{subfiles}

\begin{document}

% ? TODO: przenoszenie fizycznego sprzętu do chmury, centralizacja zasobów, łatwiejszy dostęp do zasobów, nw czy potrzebne

Można aktualnie zaobserwować dużą zmianę w rynku pracy. Z powodu globalnej epidemii wiele firm zdecydowało się na zmianę pracy stacjonarnej na zdalną. Nawet po złagodzeniu obostrzeń, znaczna część miejsc pracy pozostała przy takim trybie, lub przyjęło hybrydową formę pracy. Taka forma pracy prowadzi jednak do pewnych utrudnień. Pracownicy mogą musieć łączyć się za pomocą funkcji zdalnego pulpitu z komputerami znajdującymi się w biurze. Może to wynikać z niewystarczającej wydajności sprzętu pracownika, lub dostępu do specyficznych programów lub zasobów. W takim wypadku komputer, z którym łączy się pracownik, musi być uruchomiony, a w przypadku awarii - zrestartowany. Dodatkowo taki dostęp może być wymagany przez ograniczony czas, co powoduje, że dużą część czasu spędza włączony, ale nieużywany.

Możliwym sposobem na złagodzenie tego problemu jest użycie zmniejszonej liczby komputerów, które mogą być używane przez większą liczbę pracowników jednocześnie, za pośrednictwem maszyn wirtualnych. Tym zmniejszamy liczbę działających maszyn, a zarządzanie może być rozwiązane za za pomocą zdalnego operowania komputerem, na którym działają.

System stworzony w ramach tej pracy adresuje opisany problem. Rozwiązanie opiera się na tym wcześniej opisanym, jednocześnie rozbudowując je w sposób ułatwiający użytkowanie oraz zarządzanie.

\end{document}