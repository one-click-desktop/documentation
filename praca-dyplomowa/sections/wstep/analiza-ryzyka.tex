\documentclass[../wstep.tex]{subfiles}

\begin{document}

\begin{table}[H]
    \caption[Analiza ryzyka]{Analiza ryzyka}
    \label{risk-analysis}
    \centering
    \begin{tabular}{| p{0.4725\textwidth} | p{0.4725\textwidth} |}
        \hline
        Mocne strony
        \begin{itemize}
            \item Łatwa skalowalność pod względem liczby sesji w~systemie.
            \item Wiele rozwiązań Open Source.
            \item Elastyczność pod względem konfiguracji.
            \item Tańsze rozwiązanie niż kupno stacji roboczych.
        \end{itemize}
         &
        Słabości
        \begin{itemize}
            \item System trudny w~konfiguracji.
            \item Potrzeba wymiany sprzętu komputerowego.
            \item Krótki czas rozwoju systemu.
            \item Ograniczone doświadczenie twórców systemu.
            \item Małe prawdopodobieństwo dalszego wsparcia projektu po zakończeniu prac.
        \end{itemize}
        \\ \hline

        Okazje
        \begin{itemize}
            \item Grupa docelowa to firmy z~dużą ilością stacji roboczych.
            \item Zwiększenie zapotrzebowania na prace zdalną na rynku pracy.
        \end{itemize}
         &

        Zagrożenia
        \begin{itemize}
            \item Istnienie konkurencji ugruntowanej na rynku.
            \item System w~dużej mierze oparty o~oprogramowanie rozwijane przez inne organizacje.
        \end{itemize}
        \\ \hline
    \end{tabular}
\end{table}

\subsection{Omówienie zagrożeń}

\begin{itemize}
    \item System trudny w~konfiguracji - wysoko prawdopodobne

          Można temu zaradzić poprzez udostępnienie dokładnej dokumentacji lub ścisłą współpracę z~klientem przy wdrażaniu systemu. \\
          Waga: duża
    \item Potrzeba wymiany sprzętu komputerowego - średnio prawdopodobne

          Klient może potrzebować wymienić aktualne stacje robocze na terminale oraz zainwestować w~sprzęt serwerowy. Jednak gdy klientami będą firmy, które mają dużo pracowników pracujących spoza biura, lub dopiero tych pracowników pozyskują, to kupno terminali i~serwerów powinno być bardziej zachęcające niż kupno stacji roboczych.\\
          Waga: średnia.
    \item Krótki czas rozwoju systemu - wysoko prawdopodobne

          Czas rozwoju systemu jest bardzo ograniczony. Aby pomimo tego ograniczenia działał on w~sposób akceptowalny powinniśmy skupić się na dobrym przedyskutowaniu i~opisaniu kluczowych modułów systemu. W~czasie projektu należy pilnować aby nie dodawać nadmiarowych funkcjonalności do systemu. W~czasie implementacji konieczne będzie dokładne zaplanowanie aplikacji pod kątem testowania automatycznego. Ułatwi to wyłapywanie prostych błędów jeszcze we wczesnej fazie projektu.\\
          Waga: wysoka
    \item Ograniczone doświadczenie twórców systemu - pewne

          Jedynym sposobem na ograniczenie ryzyka jest rozważna implementacja.\\
          Waga: średnia
    \item Małe prawdopodobieństwo wsparcia projektu po zakończeniu prac - wysoko prawdopodobne

          Trudno teraz przewidzieć co się stanie z~projektem po zakończeniu prac. Jednak prawdopodobnie twórcy systemu zajmą się innymi projektami. Można jedynie dokładnie komentować kod i~pokrywać jak najwięcej jego części testami. Wtedy inne osoby będą w~stanie szukać błędów albo próbować w~taki sposób uzupełnić brakującą wiedzę o~systemie.\\
          Waga: niska
    \item Istnienie konkurencji ugruntowanej na rynku - bardzo prawdopodobne

          Konkurencyjne systemy oferujące podobne rozwiązania są już dobrze ugruntowane na rynku i~przetestowane. Nasz system może spróbować konkurować jedynie z~nimi ceną implementacji oraz elastycznością.\\
          Waga: średnia
    \item System w~dużej mierze oparty o~oprogramowanie rozwijane przez inne organizacje - nisko prawdopodobne

          W~czasie życia systemu mogą pojawić się błędy w~oprogramowaniu nie rozwijanym w~ramach naszego systemu. Naprawa takich błędów może trwać bardzo długo. Pewnym sposobem rozwiązania takiego problemu jest własnoręczne poprawianie błędów w~zewnętrznym oprogramowaniu i~zgłaszanie ich do odpowiednich organizacji. Do czasu zastosowania poprawki jest możliwość korzystania z~wersji, na którą nanieśliśmy własną poprawkę.\\
          Waga: wysoka

\end{itemize}

\end{document}