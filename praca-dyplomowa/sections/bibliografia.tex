\documentclass[../praca-dyplomowa.tex]{subfiles}

\begin{document}
%Ogolnie tutaj mozna cos ogarnac aby podzielic zrodla na zewnetrzne i wewnetrzne (nasze repa glownie)
\cite{test}
\cite{ext_test}	

\bibliography{../ocd}{}
\bibliographystyle{plain}

\bibliography{../ext}{}
\bibliographystyle{plain}
	
\iffalse
\begin{thebibliography}{20}%jak ktoś ma więcej książek, to niech wpisze większą liczbę
    % \bibitem[numerek]{referencja} Autor, \emph{Tytuł}, Wydawnictwo, rok, strony
  % cytowanie: \cite{referencja1, referencja 2,...}

  \bibitem[1]{Ktos} A. Author, \emph{Title of a book}, Publisher, year, page--page.
  \bibitem[2]{Innyktos} J. Bobkowski, S. Dobkowski, Jak stworzyć bibliografię w BibTeX-u, \emph{Czasopismo nr}, rok, strona--strona.
  \bibitem[3]{B} C. Brink, Power structures, \emph{Algebra Universalis 30(2)}, 1993, 177--216.
  \bibitem[4]{H} F. Burris, H. P. Sankappanavar, \emph{A Course of Universal Algebra}, Springer-Verlag, Nowy Jork, 1981.
\end{thebibliography}

\begin{thebibliography}{20}%jak ktoś ma więcej książek, to niech wpisze większą liczbę
	% \bibitem[numerek]{referencja} Autor, \emph{Tytuł}, Wydawnictwo, rok, strony
	% cytowanie: \cite{referencja1, referencja 2,...}
	
	\bibitem[1]{Ktos} A. Author, \emph{Title of a book}, Publisher, year, page--page.
	\bibitem[2]{Innyktos} J. Bobkowski, S. Dobkowski, Jak stworzyć bibliografię w BibTeX-u, \emph{Czasopismo nr}, rok, strona--strona.
	\bibitem[3]{B} C. Brink, Power structures, \emph{Algebra Universalis 30(2)}, 1993, 177--216.
	\bibitem[4]{H} F. Burris, H. P. Sankappanavar, \emph{A Course of Universal Algebra}, Springer-Verlag, Nowy Jork, 1981.
\end{thebibliography}
\fi

\end{document}