\documentclass[../podsumowanie.tex]{subfiles}

\begin{document}

\subsection{Konfiguracja połączenia RabbitMQ}

Klient RabbitMQ, używany przez moduły wewnętrzne do połączenia z instancją brokera, udostępnia wiele opcji konfiguracji, które nie są udostępniane przez moduły. Konfiguracja ta definiuje opcje połączenia, takie jak hasło autoryzacji, czy niestandardowe ścieżki połączenia. W przypadku korzystania z niestandardowej instancji brokera konfiguracja ta może okazać się niezbędna. Z tego powodu powinno być możliwe przekazanie ustawień połączenia z brokerem wiadomości w plikach konfiguracyjnych modułów wewnętrznych.

\subsection{Monitorowanie stanu połączenia klienta}

Do monitorowania stanu połączenia z maszyną wirtualną wykorzystywany jest broker wiadomości RabbitMQ. Klient uznawany jest za połączonego gdy istnieje kolejka z nazwą odpowiadającą identyfikatorowi sesji. Rozwiązanie to jest bardzo niestandardowe, kolejki używane są w nie zamierzony sposób. Dodatkowo klient RabbitMQ nie udostępnia możliwości powiadamiania o braku kolejki w sposób pasywny. Implementacja ta wymaga działania instancji brokera, która jest dostępna spoza wnętrza systemu. Może to powodować luki w bezpieczeństwie. Pożądane byłoby zastąpienie tej implementacji mechanizmem przeznaczonym do tego typu rozwiązań.

\subsection{Znane problemy wyścigów}

Pomimo przywiązania dużej uwagi do zaprojektowania komunikacji i procesów w sposób niwelujący ilość możliwych problemów wynikających z konkurencji wciąż istnieją miejsca, w których mogą one wystąpić. Znanym problemem jest możliwość uruchomienia dwa razy więcej maszyn niż zamierzono podczas startu systemu. Spowodowane jest to tym, że nadzorca otrzymując informację o dostępności serwera wirtualizacji poprosi go o utworzenie wymaganych maszyn wirtualnych. Jeżeli zanim serwer prześle informację o wykonaniu zadania, drugi serwer wirtualizacji również stanie się dostępny, nadzorca poprosi go o utworzenie tych samych maszyn. Taki ciąg wydarzeń doprowadzi do działania dwa razy większej liczby maszyn niż było zamierzone.

Możliwym rozwiązaniem problemu może być przeprojektowanie procesu uruchamiania wymaganych maszyn w celu uwzględnienia możliwości pojawienia się kolejnych instancji serwera wirtualizacji.

\subsection{Brak możliwości dodawania parametrów do wywołania klienta RDP}

Pewnym problemem związanym z wybraniem konkretnego klienta RDP i integracji z nim jest brak możliwości personalizacji połączenia.
W czasie testów okazało się, że niektórzy użytkownicy chcieliby zmienić rozmiar okienka klienta FreeRDP, ponieważ nie mieściło im się na ekranie.
Niestety w aktualnej implementacji użytkownik nie ma wpływu na parametry uruchomienia FreeRDP.
Prostym i wydajnym rozwiązaniem jest możliwość zdefiniowania dodatkowych parametrów uruchomienia przez użytkownika z poziomu ustawień.
FreeRDP posiada liczne parametry do konfiguracji \parencite{freerdp-manual} uruchomienia, co powinno rozwiązać problem rozmiaru okienka.

\subsection{Niezgodność stanu maszyny w systemie po interakcji z użytkownikiem}

Aktualnie system zakłada, że tylko on może zmienić stan maszyny wirtualnej.
Jakiekolwiek wyłączenie maszyny przez użytkownika lub administratora wywoła niespójność modelu.
W najgorszym wypadku może spowodować to permanentne zajęcie zasobów w systemie i konieczność restartu serwera wirtualizacji, na którym doszło do błędu.
W bibliotece libvirta od IDNT istnieje możliwość śledzenia stanów maszyn wirtualnych zaprezentowanym w dokumentacji libvirta \parencite{libvirt-vm-states}.
W przypadku przejścia maszyny do innego stanu niż \textit{Running} możemy odpowiednio zmodyfikować nasz model odzwierciedlając rzeczywisty stan maszyn wirtualnych.
Konkretna implementacja tej zmiany wymaga jeszcze dokładnego przemyślenia.

\subsection{Odporność systemu na ataki siłowe}

W tym momencie system jest nie odporny na ataki siłowego zgadnięcia hasła.
Przyjmie on dowolnie dużo zapytań, jeżeli pozwolą mu na to zasoby.
Prostym i wydajnym sposobem będzie wypisywanie informacji o nieudanym logowaniu do procesu sysloga.
Jeżeli w tym wpisie będzie zawarta informacja o adresie wysyłającego zapytania, to można zastosować system fail2ban (\url{https://www.fail2ban.org/wiki/index.php/Main_Page}).
Zablokuje on z poziomu zapory sieciowej adres, który wysyła zbyt dużo błędnych prób logowania.

\subsection{Publiczny dostęp do systemu}

System w aktualnym stanie zakłada, że klienci są w stanie skomunikować się z maszynami wirtualnymi za pomocą ich adresów lokalnych.
Jeżeli system miałby być dostępny publicznie, to należy umożliwić użytkownikom dostęp do sieci, w której działają maszyny wirtualne, przy pomocy protokołu VPN.
Problemem jest przekazywanie do aplikacji klienckiej adresów lokalnych maszyn wirtualnych.
Aby rozszerzyć jego działanie poza siec lokalną  system musiałby posiadać jakąś wiedzę na temat tłumaczenia adresów lokalnych na publiczne.
Pewna możliwością byłoby przechwytywanie odpowiedzi z adresami maszyn przez zewnętrzną aplikację znająca takie tłumaczenia.
Jednak zwiększyłaby ona złożoność systemu.

\end{document}
