\documentclass[../podsumowanie.tex]{subfiles}

\begin{document}

\subsection{Konfiguracja połączenia RabbitMQ}

Klient RabbitMQ używany przez moduły wewnętrzne do połączenia z instancją brokera udostępnia wiele opcji konfiguracji, które nie są udostępniane przez moduły. Konfiguracja ta definiuje opcje połączenia, takie jak hasło autoryzacji, czy niestandardowe ścieżki połączenia. W przypadku korzystania z niestandardowej instancji brokera, konfiguracja ta może okazać się niezbędna. Z tego powodu powinno być możliwe przekazanie ustawień połączenia z brokerem wiadomości w plikach konfiguracyjnych modułów wewnętrznych.

\subsection{Monitorowanie stanu połączenia klienta}

Do monitorowania stanu połączenia z maszyną wirtualną wykorzystywany jest broker wiadomości RabbitMQ. Klient uznawany jest za połączonego gdy istnieje kolejka z nazwą odpowiadającą identyfikatorowi sesji. Rozwiązanie to jest bardzo niestandardowe, kolejki używane są w nie zamierzony sposób. Dodatkowo klient RabbitMQ nie udostępnia możliwości powiadamiania o braku kolejki w sposób pasywny. Implementacja ta wymaga działania instancji brokera, która jest dostępna spoza wnętrza systemu. Może to powodować luki w bezpieczeństwie. Pożądane byłoby zastąpienie tej implementacji mechanizmem przeznaczonym do tego typu rozwiązań.

\subsection{Znane problemy wyścigów}

Pomimo przywiązania dużej uwagi do zaprojektowania komunikacji i procesów w sposób niwelujący ilość możliwych problemów wynikających z konkurencji, wciąż istnieją miejsca, w których mogą one wystąpić. Znanym problemem jest możliwość uruchomienia dwa razy więcej maszyn niż zamierzono podczas startu systemu. Spowodowane jest to tym, że nadzorca otrzymując informację o dostępności serwera wirtualizacji poprosi go o utworzenie wymaganych maszyn wirtualnych. Jeżeli zanim serwer prześle informację o wykonaniu zadania, drugi serwer wirtualizacji również stanie się dostępny, nadzorca poprosi go o utworzenie tych samych maszyn. Taki ciąg wydarzeń doprowadzi do działania dwa razy większej liczby maszyn niż było zamierzone.

Możliwym rozwiązaniem problemu może być przeprojektowanie procesu uruchamiania wymaganych maszyn, w celu uwzględnienia możliwości pojawienia się kolejnych instancji serwera wirtualizacji.

\end{document}
