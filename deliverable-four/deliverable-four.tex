\documentclass[12pt]{article}

\usepackage{amsmath}
\usepackage{amssymb}
\usepackage{amsfonts}
\usepackage{polski}
\usepackage{verbatim}
\usepackage[utf8]{inputenc}
\usepackage[polish]{babel}
\usepackage[T1]{fontenc}
\usepackage{graphicx}
\usepackage{caption}
\usepackage{enumitem}
\usepackage{hyperref}
\usepackage{natbib}
\usepackage{graphicx}
\usepackage{datetime}
\usepackage{multirow}
\usepackage{array}
\newdateformat{mydate}{\twodigit{\THEDAY}.\twodigit{\THEMONTH}.\THEYEAR r.}
\usepackage{enumitem}
\usepackage{footnote}
\setlist{
	noitemsep,
	listparindent=\parindent,
	parsep=0pt,
}
\makesavenoteenv{tabular}
\makesavenoteenv{table}

\usepackage{xparse}

\textheight 23.2 cm
\textwidth 6.0 in
\hoffset = -0.5 in
\voffset = -2.4 cm

\begin{document}

\begin{titlepage}
	\begin{flushright}
		{\mydate\today}\\
	\end{flushright}
	\vskip30ex

	\begin{center}
		\Large {\bf{
				System do zdalnej pracy w środowisku graficznym wykorzystujący maszyny wirtualne QEMU z akceleracja sprzętową\\
			}}
		\vskip2ex
		\bf{Sprawozdanie z testów\\}
		\vskip2ex
		\small { Autorzy: Krzysztof Smogór, Piotr Widomski\\  }
		\small { Promotor: Dr inż. Marek Kozłowski\\ }
	\end{center}
\end{titlepage}

\tableofcontents

\newpage

\section{Testy jednostkowe}

\section{Testy integracyjne}

\subsection {Testy integracji z libvirtem oraz Vagrantem}
Przy integracji systemu z libvirtem oraz vagrantem musielismy sprawdzic następujące funkcjonalności:
\begin{enumerate}
	\item Włączanie maszyn poprzez vagranta
	\item Wyłączanie maszyn poprzez vagranta
	\item Sprawdzanie czy maszyna jest uruchomiona poprzez libvirta
	\item Pobranie adresu IP uruchomionej maszyny wirtualnej
\end{enumerate}

Aby upewnić się, że wszystko działa prawidłowo skorzystaliśmy z mechaniki testów jednostkowych NUnit przy jednocześnie uruchomionym daemonie libvirta.

\section{Testy E2E}

\section{Scenariusze akceptacyjne}

\section{Aktualny stan testów}
\end{document}